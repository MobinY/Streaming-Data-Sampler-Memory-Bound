\documentclass[10pt]{article}
\usepackage{fullpage}
\usepackage{amsmath,amsfonts,amssymb,amsthm}
\usepackage{algorithm,algpseudocode,float}
\usepackage{xspace}
\usepackage[usenames]{color}
%\usepackage{hyperref}
\newcommand{\TODO}[1]{\textcolor{red}{\textbf{todo:} \textit{#1}}}

\newcommand{\supp}{\mathop{supp}}
\newcommand{\suppfind}[1]{support-finding$_{{#1}}$}

\newtheorem{theorem}{Theorem}
\newtheorem{lemma}{Lemma}
\newtheorem{definition}{Definition}
\newtheorem{corollary}{Corollary}
\newtheorem{remark}{Remark}
\DeclareMathOperator*{\E}{\mathbb{E}}
\let\Pr\relax
\DeclareMathOperator*{\Pr}{\mathbb{P}}
\newcommand{\samp}{\textsf{SAMP}\xspace}
\newcommand{\success}{\textsf{SUCC}\xspace}
\newcommand{\enc}{\textsf{ENC}\xspace}
\newcommand{\dec}{\textsf{DEC}\xspace}
\newcommand{\s}{\textsf{s}\xspace}
\newcommand{\R}{\mathbb{R}}
\newcommand{\sk}{\mathsf{sk}}
\newcommand{\sketch}{\mathsf{sketch}}
\newcommand{\query}{\mathsf{query}}
\newcommand{\eps}{\varepsilon}

\title{An optimal space lower bound for samplers}
\author{Jelani Nelson\thanks{Harvard University. \texttt{minilek@seas.harvard.edu}. Supported by NSF grant IIS-1447471 and
   CAREER award CCF-1350670, ONR Young Investigator award N00014-15-1-2388, and a Google Faculty Research Award.}
  \and Jakub Pachocki\thanks{OpenAI. \texttt{jakub.pachocki@gmail.com}. Work done while affiliated with Harvard University, under the support of ONR grant N00014-15-1-2388.}
  \and Zhengyu Wang\thanks{Harvard University. \texttt{zhengyuwang@g.harvard.edu}. Supported by NSF grant CCF-1350670.}}

\begin{document}

\maketitle

\begin{abstract}
In the $\ell_0$-sampler turnstile streaming problem, a high-dimensional vector $x\in\R^n$ receives coordinate-wise updates of the form ``$x_i\leftarrow x_i + \Delta$'', and we must  maintain a {\em sketch} $\sk(x)\in\{0,1\}^S$ so that
\begin{enumerate}
\item $S$, the memory footprint of the sketch in bits, is as small as possible, and
\item given access to only $\sk(x)$, one can recover an index $i\in[n]$ such that with probability $1-\delta$, $i$ is a uniformly random element in $\supp(x) = \{i : x_i\neq 0\}$.
\end{enumerate}
We prove that any such sketching algorithm requires $S = \Omega(\log^2 n\cdot \log(1/\delta))$ bits for any $2^{-n^{.99}} < \delta < .1$, which is nearly the full range of $\delta$ of interest and is optimal in this range \cite{JowhariST11}. Our lower bound holds even if the algorithm is allowed to output {\em any} element of the support, and not necessarily a random one, which is actually all that is needed for almost all known applications of $\ell_0$-sampling to dynamic graph streams. This furthermore means that our lower bound also holds against $\ell_p$-sampling for any $p$, and is optimal for $0<p<2$ for constant error parameter $\eps$ given the upper bound of \cite{JowhariST11}. Also, our lower bound holds even in the strict turnstile model, in which it is promised that $x_i\ge 0$ for all $i$ at all times.

We prove our lower bound by showing that $\ell_0$-samplers which are {\em too} memory-efficient can be used to compress subsets of $[n]$ of certain sizes below the information theoretic minimum. Our compression scheme makes adaptive queries to the sampler throughout its execution, but done carefully so as to not violate correctness. This is accomplished by injecting random noise into the encoder's queries to the sampler, which is loosely motivated by techniques in differential privacy. Our correctness analysis involves understanding the ability of the sampler to correctly answer adaptive queries which have non-zero, but bounded, mutual information with its internal randomness, and may be of independent interest in the newly emerging area of adaptive data analysis with a theoretical computer science lens.
\end{abstract}

\section{Introduction}
In turnstile $\ell_0$-sampling, a vector $x\in\R^n$ starts as the zero vector and receives coordinate-wise updates of the form ``$x_i \leftarrow x_i + \Delta$'' for some $\Delta\in\R$. During a query, one must return a uniformly random element from $\mathop{support}(x)$. Data structures solving this problem were first used as a subroutine to solve various dynamic graph streaming problems in \cite{AhnGM12a} and since then have been crucially used in seemingly every dynamic graph streaming problem studied across several papers: connectivity \cite{AhnGM12a}, $k$-connectivity \cite{AhnGM12a}, spanners \cite{AhnGM12b}, cut sparsifiers \cite{AhnGM12b}, spectral sparsifiers \cite{AhnGM13}, minimum spanning tree \cite{AhnGM12a}, maximal matching \cite{ChitnisCHM15}, maximum matching \cite{BuryS15,Konrad15,AssadiKLY16,ChitnisCEHMMV16,AssadiKL17}, vertex cover \cite{ChitnisCHM15,ChitnisCEHMMV16}, hitting set \cite{ChitnisCEHMMV16}, $b$-matching \cite{ChitnisCEHMMV16}, disjoint paths \cite{ChitnisCEHMMV16}, $k$-colorable subgraph \cite{ChitnisCEHMMV16}, several other maximum subgraph problems \cite{ChitnisCEHMMV16}, densest subgraph \cite{BhattacharyaHNT15,McGregorTVV15,EsfandiariHW16}, vertex and hyperedge connectivity \cite{GuhaMT15}, and approximating graph degeneracy \cite{FarachColtonT16}, to name a few.

If $x$ is $n$-dimensional with integer coordinates bounded by $\mathop{poly}(n)$, the work of \cite{JowhariST11} gave an $\Omega(\log^2 n)$-bit space lower bound for data structures which fail with constant probability, and otherwise whose query responses are close to uniform in statistical distance. They also gave an upper bound with failure probability $\delta$, which in fact gave $\min\{|\mathop{support{(x)}}|, \Theta(\log(1/\delta))\}$ uniform samples from the support of $x$, with space $O(\log^2 n \log(1/\delta))$ bits. No lower bound was known in terms of $\delta$.

We prove that for any $\ell_0$-sampling data structure, the required space complexity is $\Omega(t\cdot \log^2(n/t))$ bits, where $t = \max\{k, \log(1/\delta)\}$, to recover $k$ samples from $\mathop{support}(x)$ with failure probability $\delta$. This is optimal for all settings of $k, \delta, n$ (as long as, say, $t\cdot \log^2(n/t)\le n^{.99}$ --- there is always a trivial $O(n \log n)$-bit algorithm by storing $x$ in memory explicitly). Furthermore, our lower bound holds even if the data structure is only required to output {\em any} $k$ elements from the support, as opposed to nearly uniform ones. The previous lower bound did not hold against this weaker guarantee, despite that the fact that this weaker guarantee is actually all that is needed for most dynamic graph streaming applications mentioned above! Furthermore, since our lower bound holds regardless of which support elements the data structure finds, it holds against $\ell_p$ samplers for every $p$ (or even other distributions).


%%%
\bigskip
\bigskip
\bigskip

We study the space lower bound for maintaining a sampler over a turnstile stream. An $\ell_p$-sampler with failure probability at most $\delta$ is a randomized data structure for maintaining vector $x\in \mathbb{R}^n$ (initially 0) under a stream of updates in the form of $(i, \Delta)$ (meaning that $x_i \leftarrow x_i+\Delta$); in the end, with probability at least $1-\delta$, it gives an ``$\ell_p$-sample'' according to $x$: namely, item $i$ is sampled with probability $\frac{|x_i|^p}{\sum_{j\in [n]}{|x_j|^p}}$. 

Note that updates are independent of the randomness used in the sampler. That is, for the purpose of proving a lower bound, we assume an oblivious adversary. 

To the best of our knowledge, the best space upper bound for $\ell_0$ sampler is $O(\log^2 n \log \frac{1}{\delta})$ bits, while the previous best lower bound is $\Omega(\log^2 n +\log\frac{1}{\delta})$ bits (where $\Omega(\log^2 n)$ is shown in \cite{JowhariST11}). The bound is tight for constant $\delta$, while for example, when $\delta=\frac{1}{n}$, the gap is $\log n$. 

We show space lower bounds for maintaining a sampler for a {\em binary} vector. That is, at any time, we are guaranteed that $x\in \{0,1\}^n$. This makes our result strong in the sense that (1) the lower bound applies for any $p$; (2) the lower bound also works for strict turnstile streams.

The lower bounds are based on communication complexity in the public random coin model. Alice wants to send Bob a uniform random set $A\subseteq [n]$ of size $m$ (Bob knows $m$, but the random source generating $A$ is independent of the random source accessible to Bob). The one-way communication problem is: Alice sends some message to Bob, and Bob is required to recover $A$ completely. Since the randomness in $A$ contains $\log (^n_m)$ bits of information, any randomized protocol that works with probability $1$ requires at least $\log (^n_m)$ expected bits. 

We consider the following protocol. Alice attaches (the memory of) a sampler \samp in the message. The sampler uses public random coins as its random source, so that the sampler will behave the same at Alice's and Bob's as long as the updates are all the same. Alice will insert all the items in $A$ into \samp and send \samp to Bob. In addition, Alice will send a subset $B\subseteq A$ to Bob, so that together with $B$ and \samp, Bob is able to recover $A$ with good probability based on some protocol they have agreed on. 
 
Now we turn the previous protocol into a new one without any failure. Let \success denote the event (or a subset of the event) that Bob successfully recovers $A$ (note that Alice can simulate Bob, so she knows exactly when \success happens). If \success happens Alice will send Bob a message starting with a $1$, followed by (the memory of) \samp, then followed by the native encoding (explained later) of $B$; otherwise, Alice will send a message starting with a $0$, followed by the native encoding of $A$. We say the native encoding of a set $S\subseteq [n]$ to be an integer (expressed in binary) in $[{n \choose |S|}]$ together with $|S|$ (taking $\log n$ bits). We drop the size of the set if it is known by the receiver.

\begin{lemma} \label{lemma:lb-meta}
  Let $\s$ denote the space (in bits) used by a sampler with failure probability at most $\delta$. Let $\s'$ denote the expected number of bits to represent $B$ conditioned on \success (if we need to send some extra auxiliary information, we will also count it into $\s'$). We have 
  
  \begin{align}
  (1+\s+\s')\cdot \Pr(\success)+(1+\log(^n_m) \cdot (1-\Pr(\success)) \ge \log (^n_m).
  \end{align} 
  
  If $\Pr(\success)\ge 1/2$, we have 
  
  \begin{align} \label{eqn:lb-meta}
  \s\ge \log (^n_m) - \s' - 2.
  \end{align} 
\end{lemma}

We consider the range of failure probability $\delta$ to be 
\begin{align} \label{eqn:delta-range}
2^{-n^{c_1}}<\delta<c_2,
\end{align}
where $c_1=0.9$ and $c_2=2^{-10}$. In fact, our lower bound applies for the range of $\delta$ where $c_1$ and $c_2$ are any constants smaller than $1$. 

In Section~\ref{sec:simple-lb} we give a lower bound of $\Omega(\log n \log \frac{1}{\delta})$ bits. This illustrates some key ideas of our framework. Then we show a lower bound of $\Omega(\log^2 n \log \frac{1}{\delta})$ bits in Section~\ref{sec:optimal-lb}.
We extend our results in Section~\ref{sec:k-samples-lb} for samplers that obtain $k$ samples with failure probability at most $\delta$, and get a lower bound of $\Omega((1-\delta)k\log^2 n)$ bits, where $1\le k \le n^{.99}$ and $0\le \delta \le 1-\frac{1}{k\log n}$.

\begin{remark}
  Because the space lower bound in this note is proven via communication complexity under public random coin model, it also applies to non-uniform models of computation such as circuits and branching programs.  
\end{remark}

\begin{remark}
  The space lower bound in this note still applies if the sampler is required to output an arbitrary item whose coordinate is non-zero instead of a uniformly random one. 
\end{remark}
\section{$\Omega(\log \frac{1}{\delta} \log {\frac{n}{\log(1/\delta)}})$ Bits Lower Bound}\label{sec:simple-lb}
Let $m=\frac{1}{2}\log \frac{1}{\delta}$, namely, Alice wants to send a uniform random set $A\subseteq [n]$ of size $m$ to Bob.

\begin{algorithm}[H]
\caption{Alice's Encoder.}
\begin{algorithmic}[1]
\Procedure{$\enc_1$}{$A$}
  \State \Return $\sketch(\chi_A)$ 
\EndProcedure
\end{algorithmic}
\end{algorithm}

\begin{algorithm}[H]
\caption{Bob's Decoder.}
\begin{algorithmic}[1]
\Procedure{$\dec_1$}{$\sk$}
  \State $S\leftarrow \emptyset$
  \For {$i=1,2,\ldots,m$}
    \State $s_i\leftarrow \query(\sk, \chi_S)$
    \State $S \leftarrow S \cup \{s_i\}$
  \EndFor
  \State \Return $S$ 
\EndProcedure
\end{algorithmic}
\end{algorithm}

\begin{lemma}
  For any $A\subseteq [n]$, where $|A|=m=\frac{1}{2}\log \frac{1}{\delta}$ and $\delta\le \frac{1}{4}$, $\Pr(\dec_1(\enc_1(A))=A)\ge 1/2$.
\end{lemma}

\begin{proof}
  Let $E_S$ where $S\subset A$ denote the event that $\query(\sketch(\chi_A), \chi_S)\in A\backslash S$. We have 
  
  \begin{align}
  \Pr(\dec_1(\enc_1(A))=A)
  \ge \Pr(\bigcap_{S\subset A}{E_S})
  \ge 1 - \sum_{S\subset A}{\Pr({\overline {E_S}})} 
  \ge 1 - \delta \cdot 2^{\frac{1}{2}\log \frac{1}{\delta}}
  \ge 1/2.
  \end{align}
\end{proof}

\begin{lemma}
  $\s = \Omega(\log \frac{1}{\delta} \log \frac{n}{\log (1/\delta)})$, for $2\le \log \frac{1}{\delta} \le n$.
\end{lemma}

\begin{proof}
  This follows from \eqref{eqn:lb-meta}, since $\log {n \choose \frac{1}{2}\log \frac{1}{\delta}}=\Omega(\log \frac{1}{\delta} \log \frac{n}{\log (1/\delta)})$ and $\s'=0$. 
\end{proof}

\section{Communication Lower Bound for $\ur^\subset$} \label{sec:optimal-lb}

%In the previous section we have shown how to extract $\Theta(\log \frac{1}{\delta})$ words of information from the sketch by sequentially peeling off elements. 
%Because the number of bad events we need to union-bound increases exponentially, the approach cannot obtain more elements. 

Let $(\sketch, \query)$ denote a protocol for $\ur^\subset$ under one-way public coin model with failure probability $\delta$.
Namely, when Alice's input is $x$ and Bob's input is $y$, Alice will send $\sketch(x)$ to Bob, and Bob will output $\query(\sketch(x), y)$. 
Note that $\sketch$ and $\query$ use the same random source. 

Let us consider failure probability of $\query(\sketch(\mathbf{1}_S), \mathbf{1}_T)$ where $T\subset S$ in terms of information leak, which is defined as the mutual information (conditioned on $S$) between $T$ and random source used in the $\ur^\subset$-protocol. 
We show in this section that the failure probability is upper-bounded by information leak divided by $\log \frac{1}{\delta}$. 
Therefore, in order to invoke $\query$ with good success probability, it is sufficient to control information leak.

Our key technique is noise injection: after obtaining an element from $S\backslash T$ via the $\ur^\subset$-protocol, we not only add the element to $T$, but also randomly add $1/K$ fraction of elements in $S\backslash T$ to $T$. 
We prove that no matter how many rounds we proceed, the information leak is always bounded by $O(K)$. 
By setting $K=\Theta(\log \frac{1}{\delta})$ and $m=\sqrt{n\log\frac{1}{\delta}}$, we can proceed in $\Theta(\frac{\log (m/K)}{\log (1 +1/K)}) = \Theta(\log\frac{1}{\delta}\log\frac{n}{\log(1/\delta)})$ rounds, and thus in expectation obtain $\Theta(\log\frac{1}{\delta}\log\frac{n}{\log(1/\delta)})$ elements in $S$.
Moreover, each element contains $\Theta(\log(n/m))=\Theta(\log \frac{n}{\log (1/\delta)})$ bits of information, so we get $\Theta(\log \frac{1}{\delta}\log^2 \frac{n}{\log (1/\delta)}) $ bits of information from $\sketch(\mathbf{1}_S)$. 

\subsection{Protocol}

The parameters used by Alice and Bob are given in Algorithm~\ref{algo:para}.
Alice wants to send a uniformly random set $S$ to Bob where $|S|=m$. 
Similar to $\enc_1$, Alice computes $M \leftarrow \sketch(\mathbf{1}_S)$, and sends it to Bob. 
Moreover, Alice will send Bob a subset $B\subseteq S$ computed as follows.
Initially $B=S$ and let $A_0=S$. Alice proceeds in $R$ rounds. 
On round $r$ ($r=1,\ldots, R$), Alice computes $s_r\leftarrow \query(M, \mathbf{1}_{S\backslash A_{r-1}})$. 
Let $b$ denote a binary string of length $R$, where $b_r$ records whether $\query$ succeeds on round $r$. 
Alice will also send $b$ to Bob.
 
If $s_r\in A_{r-1}$, i.e. $\query(M, \mathbf{1}_{S\backslash A_{r-1}})$ succeeds, Alice will set $b_r=1$, and remove $s_r$ from $B$ (since $s_r$ can be obtained from the $\ur^\subset$-protocol, Alice does not need to include it in $B$. Algorithm~\ref{algo:enc} uses $D$ to keep track of $S\backslash B$); otherwise Alice will set $b_r=0$.
At the end of round $r$, Alice will generate a uniformly random set $A_r$, so that $A_r$ is a subset of $A_{r-1}\backslash \{s_r\}$ and $|A_r|=n_r$. 
In particular, Alice uses shared random source to generate $A_r$, so that Bob can recover $A_{r-1}\backslash A_r$ on round $r$. 
We present Alice's encoder in Algorithm~\ref{algo:enc}.

The decoding process is symmetric. 
Let $C_0=\emptyset$ and $D=\emptyset$. 
Bob proceeds in $R$ rounds. 
On round $r$ ($r=1,\ldots,R$), Bob obtains $s_r\in S\backslash C_{r-1}$ by invoking $\query(M, \mathbf{1}_{C_{r-1}})$. 
By the construction of $C_{r-1}$ that will be described later it is guaranteed that $A_{r-1}=S\backslash C_{r-1}$. 
Therefore, Bob will get exactly the same $s_r$ as Alice. 
Bob assigns initial value $C_{r-1}$ to $C_r$.
If $b_r=1$, Bob will add $s_r$ to both $D$ and $C_r$.
At the end of round $r$, Bob inserts a bunch of items to $C_r$ so that $C_r=S\backslash A_r$. 
Bob can achieve this because of the shared random permutation $\sigma$ when constructing $A_r$.
In the end, Bob's decoder outputs $B\cup D$.
We present the decoder in Algorithm~\ref{algo:dec}.

\begin{algorithm}[H] 
  \caption{Variables Shared by Alice's $\enc$ and Bob's $\dec$.} \label{algo:para}
  \begin{algorithmic}[1] 
    \State $m\leftarrow \lfloor \sqrt{n \log\frac{1}{\delta}} \rfloor$ \Comment{We take the floor whenever a parameter is supposed to be an integer}
    \State $K\leftarrow \lfloor \frac{1}{16}\log \frac{1}{\delta} \rfloor$
    \State $R\leftarrow \lfloor K\log(m/4K) \rfloor$
    \For {$r = 0, \ldots, R$}
      \State $n_r\leftarrow \lfloor m \cdot 2^{-\frac{r}{K}} \rfloor$ \Comment{$|A_r|=n_r$, and we have $n_r-n_{r+1}\ge 2$}
    \EndFor
    \State Let $\sigma$ be a random permutation on $[n]$ \Comment{Used to generate $A_r$ and $C_r$}
  \end{algorithmic}
\end{algorithm}

\begin{algorithm}[H] 
  \caption{Alice's Encoder.} \label{algo:enc}
  \begin{algorithmic}[1]
    \Procedure{$\enc$}{$S$}
    \State $M \leftarrow \sketch(\mathbf{1}_S)$
    \State $D\leftarrow \emptyset$
    \State $A_0 \leftarrow S$
    \For {$r=1,\ldots,R$}
      \State $s_r\leftarrow \query(M, \mathbf{1}_{S\backslash A_{r-1}})$
      \State $A_r\leftarrow A_{r-1}$
      \If {$s_r\in A_{r-1}$} \Comment{i.e. if $s_r$ is a valid sample}
        \State $b_r\leftarrow 1$ \Comment{$b$ is a binary string of length $R$, indicating if $\query$ succeeds on round $r$}
        \State $D\leftarrow D \cup \{s_r\}$
        \State $A_r\leftarrow A_r \backslash \{s_r\}$
      \Else 
        \State $b_r\leftarrow 0$
      \EndIf
      \State Remove $|A_r|-n_r$ elements from $A_r$ with smallest $\sigma_a$'s among $a\in A_r$ \Comment{So that $|A_r|=n_r$}
    \EndFor
    \State \Return ($M$, $A\backslash D$, $b$) 
    \EndProcedure
  \end{algorithmic}
\end{algorithm}

\begin{algorithm}[H] 
  \caption{Bob's Decoder.} \label{algo:dec}
  \begin{algorithmic}[1]
    \Procedure{$\dec$}{$M$, $B$, $b$}
    \State $D\leftarrow \emptyset$
    \State $C_0 \leftarrow \emptyset$
    \For {$r=1,\ldots,R$}
      \State $C_r\leftarrow C_{r-1}$
      \If{$b_r=1$}
        \State $s_r\leftarrow \query(M, \mathbf{1}_{C_{r-1}})$ \Comment{Invariant: $C_r=S \backslash A_r$ ($A_r$ is defined in $\enc$)}
        \State $D\leftarrow D \cup \{s_r\}$
        \State $C_r\leftarrow C_r \cup \{s_r\}$
      \EndIf
       \State Insert $m-n_r-|C_r|$ items into $C_r$ with smallest $\sigma_a$'s among $a\in B\backslash C_r$
    \EndFor
    \State \Return $B\cup D$ 
    \EndProcedure
  \end{algorithmic}
\end{algorithm}

\subsection{Analysis}

We have three random objects here: 
(1) Set $S$, which is a uniform random set of size $m$; 
(2) The random source used by the $\ur^\subset$-protocol, denoted by $X$; 
(3) Random permutation $\sigma$. 
They are independent. 
We do the analysis conditioned on $S$. 
Namely, the following arguments apply for any fixed $S$. 

First, we can prove that $\dec(\enc(S))=S$. 
That is, no matter the randomness in $X$ and $\sigma$, Bob will always decode $S$ successfully. 
It is because Alice's $\enc$ and Bob's $\dec$ share $X$ and $\sigma$, so that Bob essentially simulates Alice. 
We formally prove this by induction in Lemma~\ref{lemma:zero-fail-prob}. 

Now our goal is to prove that by using $\ur^\subset$-protocol, the number of bits that Alice saves is $\Omega(\log \frac{1}{\delta}\log^2 \frac{n}{\log (1/\delta)} )$. 
Intuitively, it is equivalent to prove the number of elements that Alice saves is $\Omega(\log \frac{1}{\delta}\log \frac{n}{\log (1/\delta)} )$.
We formalize this in Lemma~\ref{lemma:bits-saving}. 
Note that Alice also needs to send $b$ (i.e., whether the $\query$ succeeds on $R$ rounds) to Bob, which takes $R$ bits. 
By setting of parameters we can afford the loss of $R$ bits. 
Thus it is sufficient to prove $\E(|B|)=|S|-\Omega(\log \frac{1}{\delta}\log \frac{n}{\log (1/\delta)})$. 

We have $|S|-|B|=\sum_{r=1}^{R}b_r$. 
In Lemma~\ref{lemma:mutual-entropy-vs-fail-prob}, we prove the probability that $\query$ fails on round $r$ is upper bounded by $\frac{I(X;A_{r-1})+1}{\log \frac{1}{\delta}}$, where $I(X;A_{r-1})$ is the mutual information between $X$ and $A_{r-1}$. 
Furthermore, we will show in Lemma~\ref{lemma:mutual-entropy-bound} that $I(X;A_{r-1})$ is upper bounded by $O(K)$.
By setting of parameters we have $\E(b_r)=\Omega(1)$ and thus $\E(|S|-|B|)=\Omega(R)=\Omega(\log \frac{1}{\delta}\log \frac{n}{\log (1/\delta)})$.
 
\begin{lemma}\label{lemma:zero-fail-prob}
  $\dec(\enc(S))=S$.
\end{lemma}

\begin{proof}
  We claim that for $r=0,\ldots, R$, $\{A_r, C_r\}$ is a partition of $S$ ($A_r$ is in \enc, and $C_r$ is in \dec). We prove the claim by induction on $r$.
  
  The basis case is $r=0$. $A_0=S$ and $C_0=\emptyset$. The claim is true.
  
  Assume by induction the claim holds for $r$ ($0\le r < R$), and we consider $r+1$. 
  On round $r$, because $S\backslash A_{r-1}=C_{r-1}$, $s_r$ obtained in both sides are the same. 
  Initially $A_r=A_{r-1}$ and $C_r=C_{r-1}$, and so $\{A_r,C_r\}$ is a partition of $S$. 
  If $s_r$ is a valid sample (i.e. $s_r\in A_{r-1}$), then $b_r=1$, and \enc removes $s_r$ from $A_r$ and in the meanwhile \dec inserts $s_r$ into $C_r$, so that $\{A_r, C_r\}$ remains a partition of $S$. 
  
  After that, Alice's \enc repeats removing $a$ from $A_r$ with smallest $\sigma_a$ until $|A_r|=n_r$. 
  Symmetrically, Bob's \dec repeats inserting $a$ into $C_r$ with smallest $\sigma_a$ among $a\in B\backslash C_r$, until $|C_r|=|S|-n_r$. 
  In the end we have $|A_r|+|C_r|=|S|$, so \enc and \dec execute repetition the same number of times. 
  Moreover, we can prove that during the same repetition the element removed from $A_r$ is exactly the same element inserted to $C_r$. 
  This is because in the beginning of a repetition $\{A_r, C_r\}$ is a partition of $S$. 
  We have $B\backslash C_r\subseteq S\backslash C_r=A_r$. Let $a^*$ denote  $a\in A_r$ that minimizes $\sigma_a$. 
  We can prove that $a^*\in B\backslash C_r\subseteq A_r$ (since $a^*$ will be removed from $A_r$, it has no chance to be included in $S$ in \enc, so that $B$ contains $a^*$), and $\sigma_{a^*}$ is also the smallest among $\{\sigma_a|a\in B\backslash C_r\}$. 
  Thus in the end of the repetition, both Alice and Bob will take $a^{*}$ to update (remove from $A_r$, insert into $C_r$). 
  Therefore, $\{A_r, C_r\}$ remains a partition of $S$.
  
  Given the fact that $\{A_r, C_r\}$ is a partition of $S$, $s_r$ are the same in \enc and \dec. 
  Furthermore, $D=\{s_r|b_r=1,r=1,\ldots, R\}$ are the same in \enc and \dec.
  We know $D\subseteq S$. 
  Since \enc outputs $S\backslash D$, and \dec outputs $(S\backslash D)\cup D$, we have $\dec(\enc(S))=S$. 
\end{proof}

\begin{lemma} \label{lemma:bits-saving}
  Let $W\in \mathbb{N}$ be a random variable, and $W\le m$. 
  Moreover, $\E(W)\le m-d$. 
  We have $\E(\log {n \choose m}-\log {n \choose W})\ge d \log (\frac{n}{m}-1)$.
\end{lemma}

\begin{proof}
  \begin{align}
  \log {n \choose m}-\log {n \choose W}
  &= \log \frac{n!/(m!(n-m)!)}{n!/(W!(n-W)!)} \\
  &= \sum_{i=1}^{m-W}\log \frac{n-W-i+1}{m-i+1} \\
  &\ge (m-W)\cdot \log \frac{n-W}{m} \\
  &\ge (m-W)\cdot \log \frac{n-m}{m}
  \end{align}
  
  Taking expectation on both sides, we get $\E(\log {n \choose m}-\log {n \choose W})\ge d \log (\frac{n}{m}-1)$. 
\end{proof}

\begin{lemma}\label{lemma:mutual-entropy-vs-fail-prob}
  Consider $f$: $\{0,1\}^b\times \{0,1\}^q\rightarrow \{0,1\}$ and $X\in\{0,1\}^b$ uniformly random. If $\forall y\in \{0,1\}^q,\ \Pr(f(X,y)=1)\le \delta$ where $0<\delta<1$, then for any random variable $Y$ supported on $\{0,1\}^q$,
  \begin{align}
  \Pr(f(X,Y)=1)\le \frac{I(X;Y)+1}{\log \frac{1}{\delta}},
  \end{align}
  where $I(X;Y)$ is the mutual information (in bits) between $X$ and $Y$.
\end{lemma}

\begin{proof}
  It is equivalent to prove $I(X;Y)\ge \E(f(X,Y))\cdot \log\frac{1}{\delta}-1$. By definition of mutual entropy, $I(X;Y)=H(X)-H(X|Y)$ where $H(X)=b$ and $H(X|Y)\le 1+(1-\E(f(X,Y)))\cdot b+\E(f(X,Y))\cdot (b-\log\frac{1}{\delta})=b+1-\E(f(X,Y))\cdot \log\frac{1}{\delta}$.
  The upper bound for $H(X|Y)$ is obtained by considering the following one-way communication problem: Alice obtains both $X$ and $Y$ while Bob only gets $Y$, what is the (minimum) expected number of bits that Alice sends to Bob so that Bob can recover $X$? 
  Any protocol gives an upper bound for $H(X|Y)$, and we simply take the following protocol: first Alice sends Bob $f(X,Y)$ (taking $1$ bit); and then if $f(X,Y)=0$ Alice sends $X$ directly (taking $b$ bits), otherwise, $f(X,Y)=1$, Alice sends the index of $X$ in $\{x|f(x,Y)=1\}$ (taking $\log (\delta 2^b)=b-\log\frac{1}{\delta}$ bits).  
\end{proof}

\begin{corollary}\label{corollary:sampler-failure}
  Let $X$ denote the random source used by the $\ur^\subset$-protocol with failure probability at most $\delta$. If $S$ is a fixed set and $T\subset S$, $\Pr(\query(\sketch(\mathbf{1}_S), \mathbf{1}_T)\not\in S\backslash T)\le \frac{I(X;T)+1}{\log\frac{1}{\delta}}$.
\end{corollary}

\begin{lemma}\label{lemma:mutual-entropy-bound}
  $I(X;A_r)\le 5K$, for $r=1,\ldots, R$.
\end{lemma}

\begin{proof}
  $I(X;A_r)=H(A_r)-H(A_r|X)$. Since $|A_r|=n_r$ and $A_r\subseteq S$ where $|S|=m$, $H(A_r)\le \log {m \choose n_r}$. Now we want to lower bound $H(A_r|X)$. By definition of conditional entropy, $H(A_r|X)=\sum_x{p_x\cdot H(A_r|X=x)}$. We fix an arbitrary $x$. If we can prove that for any $T\subseteq S$ where $|T|=n_r$, $\Pr(A_r=T|X=x)\le p$, then by definition of entropy we have $H(A_r|X=x)\ge\log\frac{1}{p}$. In fact, for any fixed $T$, we have
  
  \begin{align}
    \Pr(A_r=T|X=x)\le \prod_{i=1}^{r}{\frac{{n_{i-1}-n_r-1 \choose n_{i-1}-n_i-1}}{{n_{i-1}-1 \choose n_{i-1}-n_i-1}}},
  \end{align}
  
  because on round $i$ ($1\le i \le r$), Alice removes $n_{i-1}-n_i$ elements from $A_{i-1}$ to get $A_i$. Conditioned on the event that $A_{i-1}\supseteq T$, the probability that $A_i\supseteq T$ is at most ${{n_{i-1}-n_r-1 \choose n_{i-1}-n_i-1}}/{{n_{i-1}-1 \choose n_{i-1}-n_i-1}}$, where the equation achieves when $s_i\in A_{i-1}\backslash T$, and Alice takes a uniformly random subset of $A_{i-1}\backslash \{s_i\}$ of size $n_{i-1}-n_i-1$, so that the subset does not intersect with $T$.
  
  For notation simplicity, let $n^{\underline{k}}$ denote $n\cdot (n-1)\ldots (n-k+1)$. We have 
  \begin{align}
    \prod_{i=1}^{r}{\frac{{n_{i-1}-n_r-1 \choose n_{i-1}-n_i-1}}{{n_{i-1}-1 \choose n_{i-1}-n_i-1}}}
    =\prod_{i=1}^{r}\frac{(n_{i-1}-n_r-1)!n_i!}{(n_{i-1}-1)!(n_i-n_r)!}
    =\prod_{i=1}^{r}\frac{n_i^{\underline{n_r}}}{(n_{i-1}-1)^{\underline{n_r}}}
    =\prod_{i=1}^{r} \left( \frac{n_i^{\underline{n_r}}}{n_{i-1}^{\underline{n_r}}}\cdot \frac{n_{i-1}}{n_{i-1}-n_r} \right).
  \end{align}
  
  By telescoping,
  \begin{align}
    \prod_{i=1}^{r} \frac{n_i^{\underline{n_r}}}{n_{i-1}^{\underline{n_r}}}
    =\frac{n_r^{\underline{n_r}}}{n_0^{\underline{n_r}}}
    =\frac{n_r!(n_0-n_r)!}{n_0!}=\frac{1}{{n_0 \choose n_r}}
    =\frac{1}{{m \choose n_r}}.
  \end{align}
  
  Moreover, 
  \begin{align}
    \prod_{i=1}^{r} \frac{n_{i-1}}{n_{i-1}-n_r}
    =\prod_{i=1}^{r} \frac{1}{1-2^{(i-1-r)/K}}
    =\prod_{j=1}^{r} \frac{1}{1-2^{-j/K}}
    \le \prod_{j=1}^{\infty} \frac{1}{1-2^{-j/K}}.
  \end{align}
  
  \TODO{deal with the rounding issue}
  
  By Lemma~\ref{lemma:Pochhammer}, we have $\prod_{j=1}^{\infty} \frac{1}{1-2^{-j/K}}\le 2^{5K}$. Let $p={2^{5K}}/{{m\choose n_r}}$, we have $\Pr(A_r=T|X=x)\le p$ and thus $H(A_r|X=x)\ge \log\frac{1}{p}=\log{{m\choose n_r}}-5K$. Therefore, $H(A_r|X)\ge \log{{m\choose n_r}}-5K$ and so $I(X;A_r)=H(A_r)-H(A_r|X)\le 5K$.  
\end{proof}

%By \url{http://mathworld.wolfram.com/q-PochhammerSymbol.html} We have $\prod_{j=1}^{\infty} \frac{1}{1-2^{-j/K}}\le 2^{5K}$. I think we can improve the constant 5 to 4 if we bound it more carefully. 
\begin{lemma}\label{lemma:Pochhammer}
  Let $K\in \mathbb{N}$ and $K\ge 1$. We have $\prod_{j=1}^{\infty} \frac{1}{1-2^{-j/K}}\le 2^{5K}$.
\end{lemma}

\begin{proof}
  First, we bound the product of first $2K$ terms. Note that $\frac{1}{1-2^{-x}}\le \frac{8}{3x}$ for $0<x\le 2$. Therefore, 
  \begin{align}
    \prod_{j=1}^{2K}\frac{1}{1-2^{-j/K}}
    \le (8/3)^{2K}\cdot \frac{K^{2K}}{(2K)!}
    \le (8/3)^{2K}\cdot \frac{K^{2K}}{(2K/e)^{2K}}
    = (4e/3)^{2K}
    < 2^{4K}. 
  \end{align}
  
  Then, we bound the product of the rest terms
  \begin{align}
    \prod_{j=2K+1}^{\infty}\frac{1}{1-2^{-j/K}} 
    \le \prod_{j=2K+1}^{\infty}\frac{1}{1-2^{-\lfloor j/K \rfloor}} 
    \le \prod_{i=2}^{\infty}\left( \frac{1}{1-2^{-i}}\right)^K 
    \le \left( \frac{1}{1-\sum_{i=2}^{\infty}2^{-i}}\right)^K
    = 2^K.
  \end{align}
  
  Multiplying two parts proves the lemma.
\end{proof}

\begin{theorem}
  $\randcom^{\rightarrow,pub}_\delta(\ur^\subset) = \Omega(\log \frac{1}{\delta}\log^2 \frac{n}{\log (1/\delta)} )$, given that $64 \le \log \frac{1}{\delta} \le \frac{n}{64}$.
\end{theorem}

\begin{proof}
  By Lemma~\ref{lemma:zero-fail-prob}, the success probability of protocol $(\enc,\dec)$ is $1$. 
  By Lemma~\ref{lemma:lb-meta}, we have $\s\ge \log (^n_m) - \s' -1$, where $\s'=\log n + R+ \E(\log (^n_{|B|}))$. 
  The size of $B$ is $|B|=|S|-\sum_{r=1}^{R}{b_r}$.
  By Corollary~\ref{corollary:sampler-failure}, conditioned on $S$, $\Pr(b_r=0)\le \frac{I(X;A_{r-1})+1}{\log\frac{1}{\delta}}$. 
  By Lemma~\ref{lemma:mutual-entropy-bound}, $I(X;A_{r-1})\le 5K$ (Note that when $r=1$, $I(X;A_0)=0\le 5K$). Therefore, $\E(b_r)\ge 1-\frac{5K+1}{\log\frac{1}{\delta}}$.
  By the setting of parameters (see Algorithm~\ref{algo:para}) we have $\E(b_r)\ge \frac{5}{8}$. Therefore, $\E(|B|)\le |S|-\frac{5}{8}R$. 
  By Lemma~\ref{lemma:bits-saving}, $\log (^n_m)-\E(\log (^n_{|B|}))\ge \frac{5}{8}R\cdot \log (\frac{n}{m}-1) \ge \frac{1}{2}R\log (\frac{n}{\log(1/\delta)})$. 
  Furthermore, $\frac{1}{6}R\log \frac{n}{\log (1/\delta)} \ge R$.
  Combining together we get $\s \ge \frac{R}{3}\log \frac{n}{\log(1/\delta)} -(\log n + 1)  =\Omega(\log \frac{1}{\delta}\log^2 \frac{n}{\log (1/\delta)} )$.
\end{proof}
\section{Communication Lower Bound for $\ur_k^\subset$}\label{sec:k-samples-lb}

In this section, we prove a space lower bound of $\Omega(k\log^2
\frac{n}{k})$ bits for one-way communication complexity of $\ur_k^\subset$ (where $1\le k \le \frac{n}{2^{10}}$) with failure probability at most $\delta = \frac{1}{2}$.
Let $(\sketch_k, \query_k)$ denote a $\ur_k^\subset$-protocol. 
We consider the following encoding/decoding protocol $(\enc_k, \dec_k)$ for $S\in {[n] \choose m}$. 
$\enc_k$ will compute $M\leftarrow \sketch_k(\mathbf{1}_S)$ as part of its output. 
In addition, $\enc_k$ will output $B\subseteq S$ constructed as follows.
Initially $B\leftarrow S$, and $\enc_k$ proceeds in $R=\Theta(\log (n/k))$ rounds. 
Let $A_0=S\supseteq A_1\supseteq \ldots \supseteq A_R$ where $A_r$ is generated by sub-sampling each element in $A_{r-1}$ with probability $\frac{1}{2}$. 
On round $r$ ($r=1,\ldots, R$), $\enc_k$ tries to get $k$ elements from $A_{r-1}$ by invoking $\query_k(M, \mathbf{1}_{S\backslash A_{r-1}})$, denoted by $S_k$, and removes $S_k\cap (A_{r-1}\backslash A_{r})$ (whose expected size is $\frac{k}{2}$) from $B$. 
Note that $\dec_k$ is able to recover the elements in $S_k\cap (A_{r-1}\backslash A_{r})$. 
For each round the failure probability of $\query_k$ is at most $\delta$. 
Thus we have $\E(|S|-|B|)\ge \frac{k}{2}\cdot (1-\delta) \cdot R=\Omega(k\log\frac{n}{k})$. 
Furthermore, each element contains $\Theta(\log \frac{n}{k})$ bits of information, thus yielding a lower bound of $\Omega(k\log^2\frac{n}{k})$ bits.

\subsection{Protocol}
\begin{algorithm}[H] 
  \caption{Variables Shared by Encoder $\enc_k$ and Decoder $\dec_k$.} \label{algo:para4}
  \begin{algorithmic}[1] 
    \State $m\leftarrow \lfloor \sqrt{nk} \rfloor$
    \State $R\leftarrow \lfloor \frac{1}{2}\log (n/k) - 2 \rfloor$ \Comment{Note that $R\ge 3$ because $k\le \frac{n}{2^{10}}$}
    \State $T_0\leftarrow [n]$
    \For {$r = 1, \ldots, R$}
      \State $T_r\leftarrow \emptyset$
      \State For each $a\in T_{r-1}$, $T_r\leftarrow T_r\cup \{a\}$ with probability $\frac{1}{2}$ \Comment{We have $A_r=A\cap T_r$}
    \EndFor
  \end{algorithmic}
\end{algorithm}

\begin{algorithm}[H] 
  \caption{Encoder $\enc_k$.} \label{algo:enc4}
  \begin{algorithmic}[1]
    \Procedure{$\enc_k$}{$S$}
    \State $M \leftarrow \sketch_k(\mathbf{1}_S)$
    \State $D\leftarrow \emptyset$
    \For {$r=1,\ldots,R$}
    \State $S_r\leftarrow \query_k(M, \mathbf{1}_{S\backslash (S\cap T_{r-1})})$
    \If {$S_r\subseteq A\cap T_{r-1}$} \Comment{i.e. if $S_r$ are valid}
      \State $b_r\leftarrow 1$ \Comment{$b$ is a binary string of length $R$, indicating if $\query_k$ succeeds on round $r$}
      \State $D\leftarrow D \cup (S_r\cap (T_{r-1}\backslash T_r))$
    \Else 
      \State $b_r\leftarrow 0$
    \EndIf
    \EndFor
      \State \Return ($M$, $S\backslash D$, $b$) 
    \EndProcedure
  \end{algorithmic}
\end{algorithm}

\begin{algorithm}[H] 
  \caption{Decoder $\dec_k$.} \label{algo:dec4}
  \begin{algorithmic}[1]
    \Procedure{$\dec_k$}{$M$, $B$, $b$}
    \State $D\leftarrow \emptyset$
    \State $C_0 \leftarrow \emptyset$
    \For {$r=1,\ldots,R$}
      \State $C_r\leftarrow C_{r-1}$
      \If {$b_r=1$}
        \State $S_r\leftarrow \query_k(M, \mathbf{1}_{C_{r-1}})$ \Comment{Invariant: $C_r=A\backslash (A\cap T_r)$}
        \State $D\leftarrow D \cup (S_r\cap (T_{r-1}\backslash T_r))$
        \State $C_r\leftarrow C_r \cup (S_r\cap (T_{r-1}\backslash T_r))$
      \EndIf
      \State $C_r\leftarrow C_r \cup (B\cap (T_{r-1}\backslash T_r))$
    \EndFor
    \State \Return $B\cup D$ 
    \EndProcedure
  \end{algorithmic}
\end{algorithm}

\subsection{Analysis}

\begin{theorem}
  $\randcom^{\rightarrow,pub}_\delta(\ur_k^\subset) = \Omega(k\log^2 \frac{n}{k} )$, given that $1 \le k \le \frac{n}{2^{10}}$ and $\delta \le \frac{1}{2}$.
\end{theorem}

\begin{proof}
  Let $A_r=S\cap T_r$. 
  Let $\success$ denote the event that $|S\cap T_R|=|A_R|\ge k$. 
  Note that $\E(|A_R|)=\frac{1}{2^R}m=4k$. By Chernoff bound, $\Pr(\success)\ge \frac{1}{2}$. 
  In the following, we argue conditioned on $\success$. Namely, on each round $r$, there is at least $k$ items in $A_r$.  
  
  Similar to Lemma~\ref{lemma:zero-fail-prob}, we can prove the protocol $(\enc_k,\dec_k)$ always succeeds. 
  By Lemma~\ref{lemma:lb-meta}, we have $\s\ge \log (^n_m) - \s' -2$, where $\s'=\log n + R+ \E(\log (^n_{|B|}))$. 
  The size of $B$ is $|B|=|S|-\sum_{r=1}^{R}{(b_r \cdot |S_r \cap (A_{r-1}\backslash A_r)|)}$.
  Conditioned on $S$, the randomness used by $\ur_k^\subset$-protocol is independent from $S\backslash A_{r-1}$ (for $r=1, \ldots, R$).
  Therefore, $\E(b_r)\ge 1-\delta\ge \frac{1}{2}$, and $b_r$ is independent from $|S_r \cap (A_{r-1}\backslash A_r)|$. 
  We have $\E(|S_r \cap (A_{r-1}\backslash A_r)|)=\frac{k}{2}$, and thus $\E(|S|-|B|)\ge \frac{kR}{4}$. 
  By Lemma~\ref{lemma:bits-saving}, $\log (^n_m)-\E(\log (^n_{|B|}))\ge \frac{kR}{4}\cdot \log (\frac{n}{m}-1) \ge \frac{kR}{9}\log (\frac{n}{k})$.
  Moreover, $\frac{kR}{10}\log \frac{n}{k}\ge R$.  
  Combining together we get $\s = \Omega(kR\log\frac{n}{k}) = \Omega(k\log^2 \frac{n}{k} )$.
\end{proof}


% \section*{Acknowledgments}
% We thank Vasileios Nakos for pointing out that a lower bound for strict turnstile $\ell_0$-samplers, such as ours, immediately implies the same lower bound for duplicate detection (Theorem~\ref{thm:duplicate}).

\bibliographystyle{alpha}
\bibliography{shortbib}

\end{document}
























