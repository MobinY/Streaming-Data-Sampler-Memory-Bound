\section{Introduction}
In turnstile $\ell_0$-sampling, a vector $x\in\R^n$ starts as the zero vector and receives coordinate-wise updates of the form ``$x_i \leftarrow x_i + \Delta$'' for $\Delta\in\{-M,-M+1,\ldots,M\}$. During a query, one must return a uniformly random element from $\supp(x) = \{i : x_i\neq 0\}$. The problem was first defined in \cite{FrahlingIS08}, where a data structure (or ``sketch'') for solving it was used to estimate the Euclidean minimum spanning tree, and to provide $\eps$-approximations of a point set $P$ in a geometric space (that is, one wants to maintain a subset $S\subset P$ such that for any set $R$ in a family of bounded VC-dimension, such as the set of all axis-parallel rectangles, $||R\cap S|/|S| - |R\cap P|/|P|| < \eps$). Sketches for $\ell_0$-sampling were also used to solve various dynamic graph streaming problems in \cite{AhnGM12a} and since then have been crucially used in seemingly {\em every} dynamic graph streaming algorithm, such as for: connectivity, $k$-connectivity, bipartiteness, and minimum spanning tree \cite{AhnGM12a}, subgraph counting, minimum cut, and cut-sparsifier and spanner computation \cite{AhnGM12b}, spectral sparsifiers \cite{AhnGM13}, maximal matching \cite{ChitnisCHM15}, maximum matching \cite{AhnGM12a,BuryS15,Konrad15,AssadiKLY16,ChitnisCEHMMV16,AssadiKL17}, vertex cover \cite{ChitnisCHM15,ChitnisCEHMMV16}, hitting set, $b$-matching, disjoint paths, $k$-colorable subgraph, and several other maximum subgraph problems \cite{ChitnisCEHMMV16}, densest subgraph \cite{BhattacharyaHNT15,McGregorTVV15,EsfandiariHW16}, vertex and hyperedge connectivity \cite{GuhaMT15}, and graph degeneracy \cite{FarachColtonT16}. For an introduction to the power of $\ell_0$-sketches in designing dynamic graph stream algorithms, see the recent survey of McGregor \cite[Section 3]{McGregor14}. Such sketches have also been used outside streaming, such as in distributed algorithms \cite{HegemanPPSS15,Pandurangan0S16} and data structures for dynamic connectivity \cite{KapronKM13,Wang15,GibbKKT15}.

Given the rising importance of $\ell_0$-sampling in algorithm design, a clear task is to understand the exact complexity of this problem. The work \cite{JowhariST11} gave an $\Omega(\log^2 n)$-bit space lower bound for data structures solving the case $M=1$ which fail with constant probability, and otherwise whose query responses are $(1/3)$-close to uniform in statistical distance. They also gave an upper bound for $M \le \mathop{poly}(n)$ with failure probability $\delta$, which in fact gave $\min\{\|x\|_0, \Theta(\log(1/\delta))\}$ uniform samples from the support of $x$, using space $O(\log^2 n \log(1/\delta))$ bits (here $\|x\|_0$ denotes $|\supp(x)|$). Thus we say their data structure actually solves the harder problem of $\ell_0$-sampling$_{\Theta(\log(1/\delta))}$ with failure probability $\delta$, where in $\ell_0$-sampling$_k$ the goal is to recover $\min\{\|x\|_0, k\}$ uniformly random elements, without replacement, from $\supp(x)$.  The upper and lower bounds in \cite{JowhariST11} thus match up to a constant factor for $k = 1$ and $\delta$ a constant.

\paragraph{Universal relation.} The work of \cite{JowhariST11} obtains its lower bounds for samplers (and finding duplicates) via reductions from {\em universal relation} ($\ur$). The problem $\ur$ was first defined in \cite{KarchmerRW95} and arose out of work of Karchmer and Wigderson on circuit depth lower bounds \cite{KarchmerW90}. For $f:\{0,1\}^n\rightarrow\{0,1\}$, its {\em depth} $D(f)$ is the minimum depth of a fan-in $2$ circuit over the basis $\{\neg, \vee, \wedge\}$ computing $f$. Meanwhile, the (deterministic) communication complexity $C(f)$ is defined as the minimum number of bits that need to be communicated in a correct protocol for Alice and Bob to solve the following communication problem: Alice receives $x\in f^{-1}(0)$ and Bob receives $y\in f^{-1}(1)$ (and hence in particular $x\neq y$), and they must both agree on an index $i\in[n]$ such that $x_i\neq y_i$. It is shown in \cite{KarchmerW90} that $D(f) = C(f)$, where they then used this correspondence to show a tight $\Omega(\log^2 n)$ depth lower bound on monotone circuits computing undirected $s$-$t$ connectivity. The problem $\ur$ abstracts away the function $f$ and requires Alice and Bob to agree on the index $i$ only knowing that $x,y\in\{0,1\}^n$ are unequal. The deterministic communication complexity of $\ur$ is nearly completely understood, with upper and lower bounds that match up to an additive $3$ bits, even if one requires an upper bound on the number of rounds \cite{TardosZ97}. Henceforth we also consider a generalized problem $\ur_k$, where the output must be $\min\{k, \|x-y\|_0\}$ distinct indices on which $x, y$ differ. We also use $\ur^{\subset}, \ur_k^{\subset}$ to denote the variants when promised $\supp(y)\subset \supp(x)$. Clearly $\ur, \ur_k$ can only be harder than $\ur^\subset, \ur_k^\subset$, respectively.

More than twenty years after its initial introduction in connection with circuit depth lower bounds, Jowhari et al.\ in \cite{JowhariST11} demonstrated the relevance of $\ur$ in the randomized one-way communication model for obtaining space lower bounds for certain streaming problems, such as $\ell_0$-sampling and finding duplicates in streams. In particular, if $\randcom^{\rightarrow,pub}_\delta(f)$ denotes the randomized one-way communication complexity of $f$ in the public coin model with failure probability $\delta$, \cite{JowhariST11} showed that the space complexity for finding duplicates in streams, as well as various sampling problems, with failure probability $1/3$ was at least $\randcom^{\rightarrow,pub}_{1/3}(\ur)$. They then showed $\randcom^{\rightarrow,pub}_{1/3}(\ur) = \Omega(\log^2 n)$.

The lower bound proof for $\randcom^{\rightarrow,pub}_{\delta}(\ur)$ in \cite{JowhariST11} was via reduction from a problem known as {\em augmented indexing} ($\mathbf{AugIndex}$) \cite{MiltersenNSW98}. In this problem, Alice receives $x\in\{0,1\}^n$ and Bob receives $i\in[n]$ together with $x_1,\ldots,x_{i-1}$ and Bob must output $x_i$ after receiving a single message from Alice. The work \cite{MiltersenNSW98} showed $\randcom^{\rightarrow,pub}_{1/3}(\mathbf{AugIndex}) = \Omega(n)$. This was extended to larger domains, i.e.\ in $\mathbf{AugIndex}(k)$ we have $x\in[k]^n$, in \cite{ErgunJS10,JayramW13}, where it was shown that $\randcom^{\rightarrow,pub}_{\delta}(\mathbf{AugIndex}(k)) = \Omega(n\log k)$ for any $\delta < 1 - 3/k$.

Below is our main theorem. The upper bound was almost known (it requires a very minor modification of an upper bound in \cite{JowhariST11}, which we describe in the appendix). The lower bound is proven in Sections~\ref{sec:optimal-lb} and \ref{sec:k-samples-lb}. For the remainder of this section, $t$ always denotes $\max\{k,\log(1/\delta)\}$.

\begin{theorem}\label{thm:main}
For any $\delta$ bounded away from $1$ and $1\le k\le n$, $\randcom^{\rightarrow,pub}_\delta(\ur_k) = \Theta(\min\{n, t\log^2(n/t)\})$. The lower bound holds even if promised that $\supp(y)\subset \supp(x)$.
\end{theorem}

From Theorem~\ref{thm:main} we obtain optimal lower bounds for various streaming problems quite simply. Recall in the streaming problem \textsf{FindDuplicate} there is a stream $i_1 i_2 \cdots i_m$ with each $i_j\in[n]$, and with probability $1-\delta$ we must output an $i\in[n]$ that appears at least twice in the stream (or report \textsf{Fail} if no duplicates exist), and with probability $\delta$ we may output anything. In the \textsf{FindDuplicate}$_k$ problem, the algorithm must report $\min\{k, d\}$ distinct duplicate indices, where $d$ is the actual number of duplicates.

\begin{corollary}
Any one-pass streaming algorithm for \textsf{FindDuplicate}$_k$ with failure probability $\delta$ must use space $\Omega(\min\{n, t\log^2(n/t)\})$ bits.
\end{corollary}
\begin{proof}
The proof is via reduction from $\ur_k$ with the promise that $\supp(y)\subset\supp(x)$. Suppose there were a space-$S$ algorithm $\mathcal{A}$ for \textsf{FindDuplicate}$_k$. Alice creates a stream consisting of all elements of $\supp(x)$ and runs $\mathcal{A}$ on those elements, then sends the memory contents of $\mathcal{A}$ to Bob. Bob then continues running $\mathcal{A}$ on all the elements of $[n]\backslash\supp(y)$. Then the duplicates of the resulting concatenated stream are exactly the $i$ for which $x_i\neq y_i$.
\end{proof}

\begin{corollary}
Any one-pass streaming algorithm for \suppfind{k} in the strict turnstile model must use space $\Omega(\min\{n, t\log^2(n/t)\})$, even if promised that $x\in\{0,1\}^n$ at all points in the stream.
\end{corollary}
\begin{proof}
This is again via reduction from $\ur_k$ with the promise that $\supp(y)\subset\supp(x)$. Let $\mathcal{A}$ be a space-$S$ algorithm for \suppfind{k} in the strict turnstile model. Let $z\in\R^n$ be a vector starting at zero. For each $i\in\supp(x)$, Alice sends the update $z_i \leftarrow z_i + 1$ to $\mathcal{A}$. Alice then sends the memory contents of $\mathcal{A}$ to Bob. Bob then for each $i\in\supp(y)$ sends the update $z_i\leftarrow z_i - 1$ to $\mathcal{A}$. Now note that $z$ is exactly the indicator vector of the set $\{i : x_i\neq y_i\}$.
\end{proof}

\begin{corollary}
Any one-pass streaming algorithm for $\ell_p$-sampling for any $p\ge 0$ in the strict turnstile model must use space $\Omega(\min\{n, t\log^2(n/t)\})$, even if promised $x\in\{0,1\}^n$ at all points in the stream.
\end{corollary}
\begin{proof}
This is via straightforward reduction from \suppfind{k}, since reporting $\min\{k,\|x\|_0\}$ elements of $\supp(x)$ satisfying some distributional requirements is only a harder problem than finding {\em any} $\min\{k,\|x\|_0\}$ elements of $\supp(x)$.
\end{proof}

The lower bound proven in \cite{JowhariST11} leaves four items to be better understood. First, their lower bound is not sensitive to $\delta$. Second, no lower bound is given in terms of $k$ for $\ell_0$-sampling$_k$. Third, their lower bound is proven against data structures that must be correct in the so-called {\em general turnstile} setting, which is the setting in which $x_i$ are allowed to be negative. This is in contrast to the {\em strict turnstile} setting, in which the data structure is promised that, although some updates $\Delta$ may be negative, no $x_i$ will ever be negative. The strict turnstile setting is particular relevant for dynamic graph streaming applications, since in those applications typically $x$ is indexed by $\binom{n}{2}$ for some graph on $n$ vertices, and $x_e$ is the number of copies of edge $e$ in some underlying multigraph. Edges then are never deleted unless they had previously been inserted, thus not requiring $\ell_0$-sampler subroutines that must be correct in general turnstile streams. Finally, fourth and most importantly, their lower bound is for $\ell_0$-sampling, whereas for every single application mentioned in the first paragraph (except for the two applications in \cite{FrahlingIS08}), the known algorithmic solutions using $\ell_0$-sampling as a subroutine actually need a subroutine for an {\em easier} problem, which we call {\em \suppfind{k}}. Whereas in $\ell_0$-sampling$_k$ a query asks for $k$ uniformly random elements from $\supp(x)$, in \suppfind{k} the response to a query is allowed to be a set of {\em any} $\min\{\|x\|_0, k\}$ elements from $\supp(x)$ with probability $1-\delta$, and with probability $\delta$ the response may be arbitrary. The \suppfind{k} problem is obviously easier than $\ell_0$-sampling$_k$, since the output indices from $\supp(x)$ need not be uniformly random, or even random at all. Despite this, the best known algorithm for solving \suppfind{k} with failure probability $\delta$ is to simply solve $\ell_0$-sampling$_k$ with failure probability $\delta$, which by the last paragraph is thus $O((k+\log(1/\delta))\log^2 n)$ bits. Meanwhile, from the lower bound side, no lower bound for support-finding$_k$ was known beyond the trivial $\Omega(\log\binom nk) = \Omega(k\log(n/k))$ bits required to write down a set of size $k$. This gap in our understanding is despite the fact that support-finding, not $\ell_0$-sampling, is the more relevant subroutine for most applications.


\paragraph{Our contribution:} We completely resolve all four issues discussed in the last paragraph. We show that any solution to the even the easier \suppfind{k} problem requires $\Omega(\min\{n, t \log^2(n/t)\})$ bits of space for $t = \max\{k, \log(1/\delta)\}$, even in the easier strict turnstile setting. Also, as in \cite{JowhariST11}, our lower bound does not require large weights and still holds even if it is promised that $x\in\{0,1\}^n$ at all points in the stream. Given the $O(t\log^2 n)$-bit upper bound of \cite{JowhariST11}, our lower bound is optimal for nearly the full range of $k, \delta, n$ (it is optimal as long as, say, $t\cdot \log^2(n/t)\le n^{.99}$ --- note there is always a trivial $O(n \log n)$-bit algorithm by storing $x$ in memory explicitly). Furthermore, since our lower bound holds for the easier \suppfind{} problem, it holds against $\ell_p$-samplers for every $p$ (in which case one wants to sample $i$ with probability $(1\pm\eps)|x_i|^p/\|x\|_p^p$), or more generally against any variant that specifies requirements on the output distribution within the support. Also due to upper bounds provided in \cite{JowhariST11}, our lower bound is optimal for $\ell_p$-sampling for constant $\eps$ for any $0\le p<2$. \TODO{figure out if any of the previous applications needed small $\delta$, or \suppfind{k} for $k > 1$, then say so} All our lower bounds are corollaries of a tight lower bound for a communication problem $\ur_k$, which we now proceed to describe, which turns out to also simply imply an optimal lower bound for another streaming problem: finding duplicates in streams.

% Also, as an immediate corollary, we completely resolve the complexity of finding duplicates in data streams. In this problem the stream is a sequence of integers $i_1 i_2 \cdots i_m \in [n]$, and we would like to report an index $j\in[n]$ which appeared at least twice in the stream (if no index appeared at least twice, we should return \textsf{NULL}). Again, the data structure may be randomized and behave arbitrarily with probability $\delta$. An algorithm solving this problem using $O(\log(1/\delta)\log^2 n)$ bits is given in \cite{JowhariST11}, with a lower bound of $\Omega(\log^2 n)$ bits also proven there for constant failure probability.

% \begin{theorem}\label{thm:duplicate}
% Any algorithm for finding a duplicate in a stream with failure probability $\delta$ must use space $\Omega(\min\{n, \log(1/\delta) \log^2(n / \log(1/\delta))^2\})$ bits. In particular, this is $\Omega(\log(1/\delta)\log^2 n)$ bits for $2^{-n^{.99}} \le \delta < 1$.
% \end{theorem}
% \begin{proof}
% We prove this lower bound via a simple reduction from \suppfind{1} in the strict turnstile model. Suppose the underlying vector being streamed in \suppfind{1} is $x\in\R^n$. Before processing the stream, we 
% \end{proof}


\subsection{Related work}
The question of whether $\ell_0$-sampling is possible in low memory in turnstile streams was first asked in \cite{CormodeMR05}. The first solution was given in \cite{FrahlingIS08}, where it was applied to approximate the cost of the Euclidean minimum spanning tree of a subset $S$ of a discrete geometric space subject to insertions and deletions. The algorithm given there used space $O(\log^3 n)$ bits to achieve failure probability $1/\mathop{poly}(n)$ (though it is likely that the space could be improved to $O(\log^2 n\log\log n)$ with a worse failure probability, by replacing a subroutine used there with a more recent $\ell_0$-estimation algorithm of \cite{KaneNW10}).

For $\ell_p$-sampling as defined above, the first work to realize its importance came even earlier than for $\ell_0$-sampling: \cite{CoppersmithK04} showed that an $\ell_2$-sampler using small memory would lead to a nearly space-optimal streaming algorithm for multiplicatively estimating $\|x\|_3$ in the turnstile model, but did not know how to implement such a data structure. The first implementation was given in \cite{MonemizadehW10}, where they achieved space $\mathop{poly}(\eps^{-1}\log n)$ for failure probability $1/\mathop{poly}(n)$. For $1\le p\le 2$ the space was improved to $O(\eps^{-p}\log^3 n)$ bits for constant failure probability \cite{AndoniKO11}. In \cite{JowhariST11} this bound was improved to $O(\eps^{-\max\{1,p\}}\log(1/\delta)\log^2 n)$ bits for failure probability $\delta$ when $0<p<2$ and $p\neq 1$. For $p=1$ the space bound achieved by \cite{JowhariST11} was a $\log(1/\eps)$ factor worse: $O(\eps^{-1}\log(1/\eps)\log(1/\delta)\log^2 n)$ bits.

\section{Overview of techniques}
We assume the adversary creates all stream updates in advance, oblivious to the algorithm's random bits.

As mentioned, we show space lower bounds for maintaining a \suppfind{} data structure for a {\em binary} vector. That is, our lower bound holds even if at all times in the stream we are guaranteed $x\in \{0,1\}^n$. Our lower bounds are based on communication complexity in the public random coin model. Alice wants to send Bob a uniform random set $A\subseteq [n]$ of size $m$ (Bob knows $m$, but the random source generating $A$ is independent of the random source accessible to Bob). The one-way communication problem is: Alice sends some message to Bob, and Bob is required to recover $A$ completely. Any randomized protocol that succeeds with probability $1$ requires Alice to send at least $\log (^n_m)$ bits in expectation.

We consider the following protocol. Alice attaches (the memory of) a support-finder \samp in the message. The support-finder uses public random coins as its random source, so that \samp will behave the same for Alice and for Bob as long as the updates are all the same. Alice will insert all the items in $A$ into \samp and send the memory footprint of \samp to Bob as a message. In addition, Alice will send a subset $B\subseteq A$ to Bob, so that together with $B$ and \samp, Bob is able to recover $A$ with good probability based on some protocol they have agreed on. 
 
Now we turn the previous protocol into a new one without any failure. Let \success denote the event (or a subset of the event) that Bob successfully recovers $A$ (note that Alice can simulate Bob, so she knows exactly when \success happens). If \success happens Alice will send Bob a message starting with a $1$, followed by (the memory of) \samp, then followed by a naive encoding (explained later) of $B$; otherwise, Alice will send a message starting with a $0$, followed by a naive encoding of $A$. We say the {\em naive encoding} of a set $S\subseteq [n]$ is an integer (expressed in binary) in $[{n \choose |S|}]$ together with $|S|$ (taking $\log n$ bits). We drop the size of the set if it is known by the receiver.

\begin{lemma} \label{lemma:lb-meta}
  Let $\s$ denote the space (in bits) used by a sampler with failure probability at most $\delta$. Let $\s'$ denote the expected number of bits to represent $B$ conditioned on \success (if we need to send some extra auxiliary information, we will also count it into $\s'$). We have 
  
  \begin{align}
  (1+\s+\s')\cdot \Pr(\success)+(1+\log(^n_m) \cdot (1-\Pr(\success)) \ge \log (^n_m).
  \end{align} 
  
  If $\Pr(\success)\ge 1/2$, we have 
  
  \begin{align} \label{eqn:lb-meta}
  \s\ge \log (^n_m) - \s' - 2.
  \end{align} 
\end{lemma}

% We consider the range of failure probability $\delta$ to be 
% \begin{align} \label{eqn:delta-range}
% 2^{-n^{c_1}}<\delta<c_2,
% \end{align}
% where $c_1=0.9$ and $c_2=2^{-10}$. In fact, our lower bound applies for the range of $\delta$ where $c_1$ and $c_2$ are any constants smaller than $1$. 

In Section~\ref{sec:simple-lb} we give a lower bound of $\Omega(\log(\frac n{\log(1/\delta)}) \log \frac{1}{\delta})$ bits. This illustrates some key ideas of our framework. Then we show a lower bound of $\Omega(\log^2(\frac n{\log(1/\delta)}) \log \frac{1}{\delta})$ bits in Section~\ref{sec:optimal-lb}.
We extend our results in Section~\ref{sec:k-samples-lb} to \suppfind{k} for $k\ge 1$ proving a lower bound of $\Omega(k\log^2(n/k))$ bits for constant failure probability.

\begin{remark}
\textup{
  Because our space lower bound is proven under the public coin model, it holds even against algorithms which are allowed to store hash functions (or any arbitrary number of random bits) for free.
}
\end{remark}