
%
\documentclass[conference]{IEEEtran}
\usepackage{blindtext, graphicx}

\usepackage{fullpage}
\usepackage{refcount}
\usepackage{amsmath,amsfonts,amssymb,amsthm}
\usepackage{algorithm,algpseudocode,float}
\usepackage{xspace}
\usepackage[usenames]{color}
\newcommand{\TODO}[1]{\textcolor{red}{\textbf{todo:} \textit{#1}}}
\newcommand{\jelani}[1]{\textcolor{red}{\textbf{jelani:} \textit{#1}}}

\newcommand{\supp}{\mathop{support}}
\newcommand{\suppfind}[1]{support-finding$_{{#1}}$}

\newtheorem{theorem}{Theorem}
\newtheorem{lemma}{Lemma}
\newtheorem{definition}{Definition}
\newtheorem{corollary}{Corollary}
\newtheorem{claim}{Claim}
\newtheorem{remark}{Remark}
\DeclareMathOperator*{\E}{\mathbb{E}}
\let\Pr\relax
\DeclareMathOperator*{\Pr}{\mathbb{P}}
\newcommand{\samp}{\textsf{SAMP}\xspace}
\newcommand{\success}{\textsf{SUCC}\xspace}
\newcommand{\diane}{\mathsf{Diane}}
\newcommand{\enc}{\textsf{ENC}\xspace}
\newcommand{\dec}{\textsf{DEC}\xspace}
\newcommand{\aug}{\mathbf{AugIndex}\xspace}
\newcommand{\s}{\textsf{s}\xspace}
\newcommand{\R}{\mathbb{R}}
\newcommand{\F}{\mathbb{F}}
\newcommand{\sk}{\mathsf{sk}}
\newcommand{\sketch}{\mathsf{Alice}}
\newcommand{\query}{\mathsf{Bob}}
\newcommand{\eps}{\varepsilon}
\newcommand{\ur}{\mathbf{UR}\xspace}
\newcommand{\randcom}{\mathbf{R}}
\newcommand{\findup}[1]{\textsf{FindDuplicate}$({#1})$\xspace}
\newcommand{\poly}{{\mathrm{poly}}}




% *** GRAPHICS RELATED PACKAGES ***
%
\ifCLASSINFOpdf
  % \usepackage[pdftex]{graphicx}
  % declare the path(s) where your graphic files are
  % \graphicspath{{../pdf/}{../jpeg/}}
  % and their extensions so you won't have to specify these with
  % every instance of \includegraphics
  % \DeclareGraphicsExtensions{.pdf,.jpeg,.png}
\else
  % or other class option (dvipsone, dvipdf, if not using dvips). graphicx
  % will default to the driver specified in the system graphics.cfg if no
  % driver is specified.
  % \usepackage[dvips]{graphicx}
  % declare the path(s) where your graphic files are
  % \graphicspath{{../eps/}}
  % and their extensions so you won't have to specify these with
  % every instance of \includegraphics
  % \DeclareGraphicsExtensions{.eps}
\fi


% correct bad hyphenation here
\hyphenation{op-tical net-works semi-conduc-tor}

\title{Optimal lower bounds for universal relation, and for samplers and finding duplicates in streams\thanks{This paper is a merger of \cite{NelsonPW17},
    and of work of Kapralov, Woodruff, and Yahyazadeh.}}

\begin{document}
%
% paper title
% can use linebreaks \\ within to get better formatting as desired



% author names and affiliations
% use a multiple column layout for up to three different
% affiliations
 \author{\IEEEauthorblockN{Michael Shell}
 \IEEEauthorblockA{School of Electrical and\\Computer Engineering\\
 Georgia Institute of Technology\\
 Atlanta, Georgia 30332--0250\\
 Email: http://www.michaelshell.org/contact.html}
 \and
 \IEEEauthorblockN{Homer Simpson}
 }

% conference papers do not typically use \thanks and this command
% is locked out in conference mode. If really needed, such as for
% the acknowledgment of grants, issue a \IEEEoverridecommandlockouts
% after \documentclass

% for over three affiliations, or if they all won't fit within the width
% of the page, use this alternative format:

% \author{\IEEEauthorblockN{Michael Shell\IEEEauthorrefmark{1},
% Homer Simpson\IEEEauthorrefmark{2},
% James Kirk\IEEEauthorrefmark{3}, 
% Montgomery Scott\IEEEauthorrefmark{3} and
% Eldon Tyrell\IEEEauthorrefmark{4}}
% \IEEEauthorblockA{\IEEEauthorrefmark{1}School of Electrical and Computer Engineering\\
% Georgia Institute of Technology,
% Atlanta, Georgia 30332--0250\\ Email: see http://www.michaelshell.org/contact.html}
% \IEEEauthorblockA{\IEEEauthorrefmark{2}Twentieth Century Fox, Springfield, USA\\
% Email: homer@thesimpsons.com}
% \IEEEauthorblockA{\IEEEauthorrefmark{3}Starfleet Academy, San Francisco, California 96678-2391\\
% Telephone: (800) 555--1212, Fax: (888) 555--1212}
% \IEEEauthorblockA{\IEEEauthorrefmark{4}Tyrell Inc., 123 Replicant Street, Los Angeles, California 90210--4321}}

\setcounter{page}{0}


% make the title area
\maketitle


    
\begin{abstract}
%\boldmath
In the communication problem $\ur$ (universal relation) \cite{KarchmerRW95}, Alice and Bob respectively receive $x, y \in\{0,1\}^n$ with the promise that $x\neq y$. The last player to receive a message must output an index $i$ such that $x_i\neq y_i$. We prove that the randomized one-way communication complexity of this problem in the public coin model is exactly $\Theta(\min\{n,\log(1/\delta)\log^2(\frac n{\log(1/\delta)})\})$ for failure probability $\delta$. Our lower bound holds even if promised $\mathop{support}(y)\subset \mathop{support}(x)$. As a corollary, we obtain optimal lower bounds for $\ell_p$-sampling in strict turnstile streams for $0\le p < 2$, as well as for the problem of finding duplicates in a stream. Our lower bounds do not need to use large weights, and hold even if promised $x\in\{0,1\}^n$ at all points in the stream. 

We give two different proofs of our main result. The first proof demonstrates that any algorithm $\mathcal A$ solving sampling problems in turnstile streams in low memory can be used to encode subsets of $[n]$ of certain sizes into a number of bits below the information theoretic minimum. Our encoder makes adaptive queries to $\mathcal A$ throughout its execution, but done carefully so as to not violate correctness. This is accomplished by injecting random noise into the encoder's interactions with $\mathcal A$, which is loosely motivated by techniques in differential privacy. Our correctness analysis involves understanding the ability of $\mathcal A$ to correctly answer adaptive queries which have positive but bounded mutual information with $\mathcal A$'s internal randomness, and may be of independent interest in the newly emerging area of adaptive data analysis with a theoretical computer science lens. Our second proof is via a novel randomized reduction from Augmented Indexing \cite{MiltersenNSW98} which needs to interact with $\mathcal A$ adaptively. To handle the adaptivity we identify certain likely interaction patterns and union bound over them to guarantee correct interaction on all of them. To guarantee correctness, it is important that the interaction hides some of its randomness from $\mathcal A$ in the reduction.
\end{abstract}
% IEEEtran.cls defaults to using nonbold math in the Abstract.
% This preserves the distinction between vectors and scalars. However,
% if the journal you are submitting to favors bold math in the abstract,
% then you can use LaTeX's standard command \boldmath at the very start
% of the abstract to achieve this. Many IEEE journals frown on math
% in the abstract anyway.

% Note that keywords are not normally used for peerreview papers.
\begin{IEEEkeywords}
IEEEtran, journal, \LaTeX, paper, template.
\end{IEEEkeywords}

\section{Introduction}
In turnstile $\ell_0$-sampling, a vector $x\in\R^n$ starts as the zero vector and receives coordinate-wise updates of the form ``$x_i \leftarrow x_i + \Delta$'' for some $\Delta\in\R$. During a query, one must return a uniformly random element from $\mathop{support}(x)$. Data structures solving this problem were first used as a subroutine to solve various dynamic graph streaming problems in \cite{AhnGM12a} and since then have been crucially used in seemingly every dynamic graph streaming problem studied across several papers: connectivity \cite{AhnGM12a}, $k$-connectivity \cite{AhnGM12a}, spanners \cite{AhnGM12b}, cut sparsifiers \cite{AhnGM12b}, spectral sparsifiers \cite{AhnGM13}, minimum spanning tree \cite{AhnGM12a}, maximal matching \cite{ChitnisCHM15}, maximum matching \cite{BuryS15,Konrad15,AssadiKLY16,ChitnisCEHMMV16,AssadiKL17}, vertex cover \cite{ChitnisCHM15,ChitnisCEHMMV16}, hitting set \cite{ChitnisCEHMMV16}, $b$-matching \cite{ChitnisCEHMMV16}, disjoint paths \cite{ChitnisCEHMMV16}, $k$-colorable subgraph \cite{ChitnisCEHMMV16}, several other maximum subgraph problems \cite{ChitnisCEHMMV16}, densest subgraph \cite{BhattacharyaHNT15,McGregorTVV15,EsfandiariHW16}, vertex and hyperedge connectivity \cite{GuhaMT15}, and approximating graph degeneracy \cite{FarachColtonT16}, to name a few.

If $x$ is $n$-dimensional with integer coordinates bounded by $\mathop{poly}(n)$, the work of \cite{JowhariST11} gave an $\Omega(\log^2 n)$-bit space lower bound for data structures which fail with constant probability, and otherwise whose query responses are close to uniform in statistical distance. They also gave an upper bound with failure probability $\delta$, which in fact gave $\min\{|\mathop{support{(x)}}|, \Theta(\log(1/\delta))\}$ uniform samples from the support of $x$, with space $O(\log^2 n \log(1/\delta))$ bits. No lower bound was known in terms of $\delta$.

We prove that for any $\ell_0$-sampling data structure, the required space complexity is $\Omega(t\cdot \log^2(n/t))$ bits, where $t = \max\{k, \log(1/\delta)\}$, to recover $k$ samples from $\mathop{support}(x)$ with failure probability $\delta$. This is optimal for all settings of $k, \delta, n$ (as long as, say, $t\cdot \log^2(n/t)\le n^{.99}$ --- there is always a trivial $O(n \log n)$-bit algorithm by storing $x$ in memory explicitly). Furthermore, our lower bound holds even if the data structure is only required to output {\em any} $k$ elements from the support, as opposed to nearly uniform ones. The previous lower bound did not hold against this weaker guarantee, despite that the fact that this weaker guarantee is actually all that is needed for most dynamic graph streaming applications mentioned above! Furthermore, since our lower bound holds regardless of which support elements the data structure finds, it holds against $\ell_p$ samplers for every $p$ (or even other distributions).


%%%
\bigskip
\bigskip
\bigskip

We study the space lower bound for maintaining a sampler over a turnstile stream. An $\ell_p$-sampler with failure probability at most $\delta$ is a randomized data structure for maintaining vector $x\in \mathbb{R}^n$ (initially 0) under a stream of updates in the form of $(i, \Delta)$ (meaning that $x_i \leftarrow x_i+\Delta$); in the end, with probability at least $1-\delta$, it gives an ``$\ell_p$-sample'' according to $x$: namely, item $i$ is sampled with probability $\frac{|x_i|^p}{\sum_{j\in [n]}{|x_j|^p}}$. 

Note that updates are independent of the randomness used in the sampler. That is, for the purpose of proving a lower bound, we assume an oblivious adversary. 

To the best of our knowledge, the best space upper bound for $\ell_0$ sampler is $O(\log^2 n \log \frac{1}{\delta})$ bits, while the previous best lower bound is $\Omega(\log^2 n +\log\frac{1}{\delta})$ bits (where $\Omega(\log^2 n)$ is shown in \cite{JowhariST11}). The bound is tight for constant $\delta$, while for example, when $\delta=\frac{1}{n}$, the gap is $\log n$. 

We show space lower bounds for maintaining a sampler for a {\em binary} vector. That is, at any time, we are guaranteed that $x\in \{0,1\}^n$. This makes our result strong in the sense that (1) the lower bound applies for any $p$; (2) the lower bound also works for strict turnstile streams.

The lower bounds are based on communication complexity in the public random coin model. Alice wants to send Bob a uniform random set $A\subseteq [n]$ of size $m$ (Bob knows $m$, but the random source generating $A$ is independent of the random source accessible to Bob). The one-way communication problem is: Alice sends some message to Bob, and Bob is required to recover $A$ completely. Since the randomness in $A$ contains $\log (^n_m)$ bits of information, any randomized protocol that works with probability $1$ requires at least $\log (^n_m)$ expected bits. 

We consider the following protocol. Alice attaches (the memory of) a sampler \samp in the message. The sampler uses public random coins as its random source, so that the sampler will behave the same at Alice's and Bob's as long as the updates are all the same. Alice will insert all the items in $A$ into \samp and send \samp to Bob. In addition, Alice will send a subset $B\subseteq A$ to Bob, so that together with $B$ and \samp, Bob is able to recover $A$ with good probability based on some protocol they have agreed on. 
 
Now we turn the previous protocol into a new one without any failure. Let \success denote the event (or a subset of the event) that Bob successfully recovers $A$ (note that Alice can simulate Bob, so she knows exactly when \success happens). If \success happens Alice will send Bob a message starting with a $1$, followed by (the memory of) \samp, then followed by the native encoding (explained later) of $B$; otherwise, Alice will send a message starting with a $0$, followed by the native encoding of $A$. We say the native encoding of a set $S\subseteq [n]$ to be an integer (expressed in binary) in $[{n \choose |S|}]$ together with $|S|$ (taking $\log n$ bits). We drop the size of the set if it is known by the receiver.

\begin{lemma} \label{lemma:lb-meta}
  Let $\s$ denote the space (in bits) used by a sampler with failure probability at most $\delta$. Let $\s'$ denote the expected number of bits to represent $B$ conditioned on \success (if we need to send some extra auxiliary information, we will also count it into $\s'$). We have 
  
  \begin{align}
  (1+\s+\s')\cdot \Pr(\success)+(1+\log(^n_m) \cdot (1-\Pr(\success)) \ge \log (^n_m).
  \end{align} 
  
  If $\Pr(\success)\ge 1/2$, we have 
  
  \begin{align} \label{eqn:lb-meta}
  \s\ge \log (^n_m) - \s' - 2.
  \end{align} 
\end{lemma}

We consider the range of failure probability $\delta$ to be 
\begin{align} \label{eqn:delta-range}
2^{-n^{c_1}}<\delta<c_2,
\end{align}
where $c_1=0.9$ and $c_2=2^{-10}$. In fact, our lower bound applies for the range of $\delta$ where $c_1$ and $c_2$ are any constants smaller than $1$. 

In Section~\ref{sec:simple-lb} we give a lower bound of $\Omega(\log n \log \frac{1}{\delta})$ bits. This illustrates some key ideas of our framework. Then we show a lower bound of $\Omega(\log^2 n \log \frac{1}{\delta})$ bits in Section~\ref{sec:optimal-lb}.
We extend our results in Section~\ref{sec:k-samples-lb} for samplers that obtain $k$ samples with failure probability at most $\delta$, and get a lower bound of $\Omega((1-\delta)k\log^2 n)$ bits, where $1\le k \le n^{.99}$ and $0\le \delta \le 1-\frac{1}{k\log n}$.

\begin{remark}
  Because the space lower bound in this note is proven via communication complexity under public random coin model, it also applies to non-uniform models of computation such as circuits and branching programs.  
\end{remark}

\begin{remark}
  The space lower bound in this note still applies if the sampler is required to output an arbitrary item whose coordinate is non-zero instead of a uniformly random one. 
\end{remark}
\section{Communication Lower Bound for $\ur^\subset$} \label{sec:optimal-lb}

%In the previous section we have shown how to extract $\Theta(\log \frac{1}{\delta})$ words of information from the sketch by sequentially peeling off elements. 
%Because the number of bad events we need to union-bound increases exponentially, the approach cannot obtain more elements. 

Let $(\sketch, \query)$ denote a protocol for $\ur^\subset$ under one-way public coin model with failure probability $\delta$.
Namely, when Alice's input is $x$ and Bob's input is $y$, Alice will send $\sketch(x)$ to Bob, and Bob will output $\query(\sketch(x), y)$. 
Note that $\sketch$ and $\query$ use the same random source. 

Let us consider failure probability of $\query(\sketch(\mathbf{1}_S), \mathbf{1}_T)$ where $T\subset S$ in terms of information leak, which is defined as the mutual information (conditioned on $S$) between $T$ and random source used in the $\ur^\subset$-protocol. 
We show in this section that the failure probability is upper-bounded by information leak divided by $\log \frac{1}{\delta}$. 
Therefore, in order to invoke $\query$ with good success probability, it is sufficient to control information leak.

Our key technique is noise injection: after obtaining an element from $S\backslash T$ via the $\ur^\subset$-protocol, we not only add the element to $T$, but also randomly add $1/K$ fraction of elements in $S\backslash T$ to $T$. 
We prove that no matter how many rounds we proceed, the information leak is always bounded by $O(K)$. 
By setting $K=\Theta(\log \frac{1}{\delta})$ and $m=\sqrt{n\log\frac{1}{\delta}}$, we can proceed in $\Theta(\frac{\log (m/K)}{\log (1 +1/K)}) = \Theta(\log\frac{1}{\delta}\log\frac{n}{\log(1/\delta)})$ rounds, and thus in expectation obtain $\Theta(\log\frac{1}{\delta}\log\frac{n}{\log(1/\delta)})$ elements in $S$.
Moreover, each element contains $\Theta(\log(n/m))=\Theta(\log \frac{n}{\log (1/\delta)})$ bits of information, so we get $\Theta(\log \frac{1}{\delta}\log^2 \frac{n}{\log (1/\delta)}) $ bits of information from $\sketch(\mathbf{1}_S)$. 

\subsection{Protocol}

The parameters used by Alice and Bob are given in Algorithm~\ref{algo:para}.
Alice wants to send a uniformly random set $S$ to Bob where $|S|=m$. 
Similar to $\enc_1$, Alice computes $M \leftarrow \sketch(\mathbf{1}_S)$, and sends it to Bob. 
Moreover, Alice will send Bob a subset $B\subseteq S$ computed as follows.
Initially $B=S$ and let $A_0=S$. Alice proceeds in $R$ rounds. 
On round $r$ ($r=1,\ldots, R$), Alice computes $s_r\leftarrow \query(M, \mathbf{1}_{S\backslash A_{r-1}})$. 
Let $b$ denote a binary string of length $R$, where $b_r$ records whether $\query$ succeeds on round $r$. 
Alice will also send $b$ to Bob.
 
If $s_r\in A_{r-1}$, i.e. $\query(M, \mathbf{1}_{S\backslash A_{r-1}})$ succeeds, Alice will set $b_r=1$, and remove $s_r$ from $B$ (since $s_r$ can be obtained from the $\ur^\subset$-protocol, Alice does not need to include it in $B$. Algorithm~\ref{algo:enc} uses $D$ to keep track of $S\backslash B$); otherwise Alice will set $b_r=0$.
At the end of round $r$, Alice will generate a uniformly random set $A_r$, so that $A_r$ is a subset of $A_{r-1}\backslash \{s_r\}$ and $|A_r|=n_r$. 
In particular, Alice uses shared random source to generate $A_r$, so that Bob can recover $A_{r-1}\backslash A_r$ on round $r$. 
We present Alice's encoder in Algorithm~\ref{algo:enc}.

The decoding process is symmetric. 
Let $C_0=\emptyset$ and $D=\emptyset$. 
Bob proceeds in $R$ rounds. 
On round $r$ ($r=1,\ldots,R$), Bob obtains $s_r\in S\backslash C_{r-1}$ by invoking $\query(M, \mathbf{1}_{C_{r-1}})$. 
By the construction of $C_{r-1}$ that will be described later it is guaranteed that $A_{r-1}=S\backslash C_{r-1}$. 
Therefore, Bob will get exactly the same $s_r$ as Alice. 
Bob assigns initial value $C_{r-1}$ to $C_r$.
If $b_r=1$, Bob will add $s_r$ to both $D$ and $C_r$.
At the end of round $r$, Bob inserts a bunch of items to $C_r$ so that $C_r=S\backslash A_r$. 
Bob can achieve this because of the shared random permutation $\sigma$ when constructing $A_r$.
In the end, Bob's decoder outputs $B\cup D$.
We present the decoder in Algorithm~\ref{algo:dec}.

\begin{algorithm}[H] 
  \caption{Variables Shared by Alice's $\enc$ and Bob's $\dec$.} \label{algo:para}
  \begin{algorithmic}[1] 
    \State $m\leftarrow \lfloor \sqrt{n \log\frac{1}{\delta}} \rfloor$ \Comment{We take the floor whenever a parameter is supposed to be an integer}
    \State $K\leftarrow \lfloor \frac{1}{16}\log \frac{1}{\delta} \rfloor$
    \State $R\leftarrow \lfloor K\log(m/4K) \rfloor$
    \For {$r = 0, \ldots, R$}
      \State $n_r\leftarrow \lfloor m \cdot 2^{-\frac{r}{K}} \rfloor$ \Comment{$|A_r|=n_r$, and we have $n_r-n_{r+1}\ge 2$}
    \EndFor
    \State Let $\sigma$ be a random permutation on $[n]$ \Comment{Used to generate $A_r$ and $C_r$}
  \end{algorithmic}
\end{algorithm}

\begin{algorithm}[H] 
  \caption{Alice's Encoder.} \label{algo:enc}
  \begin{algorithmic}[1]
    \Procedure{$\enc$}{$S$}
    \State $M \leftarrow \sketch(\mathbf{1}_S)$
    \State $D\leftarrow \emptyset$
    \State $A_0 \leftarrow S$
    \For {$r=1,\ldots,R$}
      \State $s_r\leftarrow \query(M, \mathbf{1}_{S\backslash A_{r-1}})$
      \State $A_r\leftarrow A_{r-1}$
      \If {$s_r\in A_{r-1}$} \Comment{i.e. if $s_r$ is a valid sample}
        \State $b_r\leftarrow 1$ \Comment{$b$ is a binary string of length $R$, indicating if $\query$ succeeds on round $r$}
        \State $D\leftarrow D \cup \{s_r\}$
        \State $A_r\leftarrow A_r \backslash \{s_r\}$
      \Else 
        \State $b_r\leftarrow 0$
      \EndIf
      \State Remove $|A_r|-n_r$ elements from $A_r$ with smallest $\sigma_a$'s among $a\in A_r$ \Comment{So that $|A_r|=n_r$}
    \EndFor
    \State \Return ($M$, $A\backslash D$, $b$) 
    \EndProcedure
  \end{algorithmic}
\end{algorithm}

\begin{algorithm}[H] 
  \caption{Bob's Decoder.} \label{algo:dec}
  \begin{algorithmic}[1]
    \Procedure{$\dec$}{$M$, $B$, $b$}
    \State $D\leftarrow \emptyset$
    \State $C_0 \leftarrow \emptyset$
    \For {$r=1,\ldots,R$}
      \State $C_r\leftarrow C_{r-1}$
      \If{$b_r=1$}
        \State $s_r\leftarrow \query(M, \mathbf{1}_{C_{r-1}})$ \Comment{Invariant: $C_r=S \backslash A_r$ ($A_r$ is defined in $\enc$)}
        \State $D\leftarrow D \cup \{s_r\}$
        \State $C_r\leftarrow C_r \cup \{s_r\}$
      \EndIf
       \State Insert $m-n_r-|C_r|$ items into $C_r$ with smallest $\sigma_a$'s among $a\in B\backslash C_r$
    \EndFor
    \State \Return $B\cup D$ 
    \EndProcedure
  \end{algorithmic}
\end{algorithm}

\subsection{Analysis}

We have three random objects here: 
(1) Set $S$, which is a uniform random set of size $m$; 
(2) The random source used by the $\ur^\subset$-protocol, denoted by $X$; 
(3) Random permutation $\sigma$. 
They are independent. 
We do the analysis conditioned on $S$. 
Namely, the following arguments apply for any fixed $S$. 

First, we can prove that $\dec(\enc(S))=S$. 
That is, no matter the randomness in $X$ and $\sigma$, Bob will always decode $S$ successfully. 
It is because Alice's $\enc$ and Bob's $\dec$ share $X$ and $\sigma$, so that Bob essentially simulates Alice. 
We formally prove this by induction in Lemma~\ref{lemma:zero-fail-prob}. 

Now our goal is to prove that by using $\ur^\subset$-protocol, the number of bits that Alice saves is $\Omega(\log \frac{1}{\delta}\log^2 \frac{n}{\log (1/\delta)} )$. 
Intuitively, it is equivalent to prove the number of elements that Alice saves is $\Omega(\log \frac{1}{\delta}\log \frac{n}{\log (1/\delta)} )$.
We formalize this in Lemma~\ref{lemma:bits-saving}. 
Note that Alice also needs to send $b$ (i.e., whether the $\query$ succeeds on $R$ rounds) to Bob, which takes $R$ bits. 
By setting of parameters we can afford the loss of $R$ bits. 
Thus it is sufficient to prove $\E(|B|)=|S|-\Omega(\log \frac{1}{\delta}\log \frac{n}{\log (1/\delta)})$. 

We have $|S|-|B|=\sum_{r=1}^{R}b_r$. 
In Lemma~\ref{lemma:mutual-entropy-vs-fail-prob}, we prove the probability that $\query$ fails on round $r$ is upper bounded by $\frac{I(X;A_{r-1})+1}{\log \frac{1}{\delta}}$, where $I(X;A_{r-1})$ is the mutual information between $X$ and $A_{r-1}$. 
Furthermore, we will show in Lemma~\ref{lemma:mutual-entropy-bound} that $I(X;A_{r-1})$ is upper bounded by $O(K)$.
By setting of parameters we have $\E(b_r)=\Omega(1)$ and thus $\E(|S|-|B|)=\Omega(R)=\Omega(\log \frac{1}{\delta}\log \frac{n}{\log (1/\delta)})$.
 
\begin{lemma}\label{lemma:zero-fail-prob}
  $\dec(\enc(S))=S$.
\end{lemma}

\begin{proof}
  We claim that for $r=0,\ldots, R$, $\{A_r, C_r\}$ is a partition of $S$ ($A_r$ is in \enc, and $C_r$ is in \dec). We prove the claim by induction on $r$.
  
  The basis case is $r=0$. $A_0=S$ and $C_0=\emptyset$. The claim is true.
  
  Assume by induction the claim holds for $r$ ($0\le r < R$), and we consider $r+1$. 
  On round $r$, because $S\backslash A_{r-1}=C_{r-1}$, $s_r$ obtained in both sides are the same. 
  Initially $A_r=A_{r-1}$ and $C_r=C_{r-1}$, and so $\{A_r,C_r\}$ is a partition of $S$. 
  If $s_r$ is a valid sample (i.e. $s_r\in A_{r-1}$), then $b_r=1$, and \enc removes $s_r$ from $A_r$ and in the meanwhile \dec inserts $s_r$ into $C_r$, so that $\{A_r, C_r\}$ remains a partition of $S$. 
  
  After that, Alice's \enc repeats removing $a$ from $A_r$ with smallest $\sigma_a$ until $|A_r|=n_r$. 
  Symmetrically, Bob's \dec repeats inserting $a$ into $C_r$ with smallest $\sigma_a$ among $a\in B\backslash C_r$, until $|C_r|=|S|-n_r$. 
  In the end we have $|A_r|+|C_r|=|S|$, so \enc and \dec execute repetition the same number of times. 
  Moreover, we can prove that during the same repetition the element removed from $A_r$ is exactly the same element inserted to $C_r$. 
  This is because in the beginning of a repetition $\{A_r, C_r\}$ is a partition of $S$. 
  We have $B\backslash C_r\subseteq S\backslash C_r=A_r$. Let $a^*$ denote  $a\in A_r$ that minimizes $\sigma_a$. 
  We can prove that $a^*\in B\backslash C_r\subseteq A_r$ (since $a^*$ will be removed from $A_r$, it has no chance to be included in $S$ in \enc, so that $B$ contains $a^*$), and $\sigma_{a^*}$ is also the smallest among $\{\sigma_a|a\in B\backslash C_r\}$. 
  Thus in the end of the repetition, both Alice and Bob will take $a^{*}$ to update (remove from $A_r$, insert into $C_r$). 
  Therefore, $\{A_r, C_r\}$ remains a partition of $S$.
  
  Given the fact that $\{A_r, C_r\}$ is a partition of $S$, $s_r$ are the same in \enc and \dec. 
  Furthermore, $D=\{s_r|b_r=1,r=1,\ldots, R\}$ are the same in \enc and \dec.
  We know $D\subseteq S$. 
  Since \enc outputs $S\backslash D$, and \dec outputs $(S\backslash D)\cup D$, we have $\dec(\enc(S))=S$. 
\end{proof}

\begin{lemma} \label{lemma:bits-saving}
  Let $W\in \mathbb{N}$ be a random variable, and $W\le m$. 
  Moreover, $\E(W)\le m-d$. 
  We have $\E(\log {n \choose m}-\log {n \choose W})\ge d \log (\frac{n}{m}-1)$.
\end{lemma}

\begin{proof}
  \begin{align}
  \log {n \choose m}-\log {n \choose W}
  &= \log \frac{n!/(m!(n-m)!)}{n!/(W!(n-W)!)} \\
  &= \sum_{i=1}^{m-W}\log \frac{n-W-i+1}{m-i+1} \\
  &\ge (m-W)\cdot \log \frac{n-W}{m} \\
  &\ge (m-W)\cdot \log \frac{n-m}{m}
  \end{align}
  
  Taking expectation on both sides, we get $\E(\log {n \choose m}-\log {n \choose W})\ge d \log (\frac{n}{m}-1)$. 
\end{proof}

\begin{lemma}\label{lemma:mutual-entropy-vs-fail-prob}
  Consider $f$: $\{0,1\}^b\times \{0,1\}^q\rightarrow \{0,1\}$ and $X\in\{0,1\}^b$ uniformly random. If $\forall y\in \{0,1\}^q,\ \Pr(f(X,y)=1)\le \delta$ where $0<\delta<1$, then for any random variable $Y$ supported on $\{0,1\}^q$,
  \begin{align}
  \Pr(f(X,Y)=1)\le \frac{I(X;Y)+1}{\log \frac{1}{\delta}},
  \end{align}
  where $I(X;Y)$ is the mutual information (in bits) between $X$ and $Y$.
\end{lemma}

\begin{proof}
  It is equivalent to prove $I(X;Y)\ge \E(f(X,Y))\cdot \log\frac{1}{\delta}-1$. By definition of mutual entropy, $I(X;Y)=H(X)-H(X|Y)$ where $H(X)=b$ and $H(X|Y)\le 1+(1-\E(f(X,Y)))\cdot b+\E(f(X,Y))\cdot (b-\log\frac{1}{\delta})=b+1-\E(f(X,Y))\cdot \log\frac{1}{\delta}$.
  The upper bound for $H(X|Y)$ is obtained by considering the following one-way communication problem: Alice obtains both $X$ and $Y$ while Bob only gets $Y$, what is the (minimum) expected number of bits that Alice sends to Bob so that Bob can recover $X$? 
  Any protocol gives an upper bound for $H(X|Y)$, and we simply take the following protocol: first Alice sends Bob $f(X,Y)$ (taking $1$ bit); and then if $f(X,Y)=0$ Alice sends $X$ directly (taking $b$ bits), otherwise, $f(X,Y)=1$, Alice sends the index of $X$ in $\{x|f(x,Y)=1\}$ (taking $\log (\delta 2^b)=b-\log\frac{1}{\delta}$ bits).  
\end{proof}

\begin{corollary}\label{corollary:sampler-failure}
  Let $X$ denote the random source used by the $\ur^\subset$-protocol with failure probability at most $\delta$. If $S$ is a fixed set and $T\subset S$, $\Pr(\query(\sketch(\mathbf{1}_S), \mathbf{1}_T)\not\in S\backslash T)\le \frac{I(X;T)+1}{\log\frac{1}{\delta}}$.
\end{corollary}

\begin{lemma}\label{lemma:mutual-entropy-bound}
  $I(X;A_r)\le 5K$, for $r=1,\ldots, R$.
\end{lemma}

\begin{proof}
  $I(X;A_r)=H(A_r)-H(A_r|X)$. Since $|A_r|=n_r$ and $A_r\subseteq S$ where $|S|=m$, $H(A_r)\le \log {m \choose n_r}$. Now we want to lower bound $H(A_r|X)$. By definition of conditional entropy, $H(A_r|X)=\sum_x{p_x\cdot H(A_r|X=x)}$. We fix an arbitrary $x$. If we can prove that for any $T\subseteq S$ where $|T|=n_r$, $\Pr(A_r=T|X=x)\le p$, then by definition of entropy we have $H(A_r|X=x)\ge\log\frac{1}{p}$. In fact, for any fixed $T$, we have
  
  \begin{align}
    \Pr(A_r=T|X=x)\le \prod_{i=1}^{r}{\frac{{n_{i-1}-n_r-1 \choose n_{i-1}-n_i-1}}{{n_{i-1}-1 \choose n_{i-1}-n_i-1}}},
  \end{align}
  
  because on round $i$ ($1\le i \le r$), Alice removes $n_{i-1}-n_i$ elements from $A_{i-1}$ to get $A_i$. Conditioned on the event that $A_{i-1}\supseteq T$, the probability that $A_i\supseteq T$ is at most ${{n_{i-1}-n_r-1 \choose n_{i-1}-n_i-1}}/{{n_{i-1}-1 \choose n_{i-1}-n_i-1}}$, where the equation achieves when $s_i\in A_{i-1}\backslash T$, and Alice takes a uniformly random subset of $A_{i-1}\backslash \{s_i\}$ of size $n_{i-1}-n_i-1$, so that the subset does not intersect with $T$.
  
  For notation simplicity, let $n^{\underline{k}}$ denote $n\cdot (n-1)\ldots (n-k+1)$. We have 
  \begin{align}
    \prod_{i=1}^{r}{\frac{{n_{i-1}-n_r-1 \choose n_{i-1}-n_i-1}}{{n_{i-1}-1 \choose n_{i-1}-n_i-1}}}
    =\prod_{i=1}^{r}\frac{(n_{i-1}-n_r-1)!n_i!}{(n_{i-1}-1)!(n_i-n_r)!}
    =\prod_{i=1}^{r}\frac{n_i^{\underline{n_r}}}{(n_{i-1}-1)^{\underline{n_r}}}
    =\prod_{i=1}^{r} \left( \frac{n_i^{\underline{n_r}}}{n_{i-1}^{\underline{n_r}}}\cdot \frac{n_{i-1}}{n_{i-1}-n_r} \right).
  \end{align}
  
  By telescoping,
  \begin{align}
    \prod_{i=1}^{r} \frac{n_i^{\underline{n_r}}}{n_{i-1}^{\underline{n_r}}}
    =\frac{n_r^{\underline{n_r}}}{n_0^{\underline{n_r}}}
    =\frac{n_r!(n_0-n_r)!}{n_0!}=\frac{1}{{n_0 \choose n_r}}
    =\frac{1}{{m \choose n_r}}.
  \end{align}
  
  Moreover, 
  \begin{align}
    \prod_{i=1}^{r} \frac{n_{i-1}}{n_{i-1}-n_r}
    =\prod_{i=1}^{r} \frac{1}{1-2^{(i-1-r)/K}}
    =\prod_{j=1}^{r} \frac{1}{1-2^{-j/K}}
    \le \prod_{j=1}^{\infty} \frac{1}{1-2^{-j/K}}.
  \end{align}
  
  \TODO{deal with the rounding issue}
  
  By Lemma~\ref{lemma:Pochhammer}, we have $\prod_{j=1}^{\infty} \frac{1}{1-2^{-j/K}}\le 2^{5K}$. Let $p={2^{5K}}/{{m\choose n_r}}$, we have $\Pr(A_r=T|X=x)\le p$ and thus $H(A_r|X=x)\ge \log\frac{1}{p}=\log{{m\choose n_r}}-5K$. Therefore, $H(A_r|X)\ge \log{{m\choose n_r}}-5K$ and so $I(X;A_r)=H(A_r)-H(A_r|X)\le 5K$.  
\end{proof}

%By \url{http://mathworld.wolfram.com/q-PochhammerSymbol.html} We have $\prod_{j=1}^{\infty} \frac{1}{1-2^{-j/K}}\le 2^{5K}$. I think we can improve the constant 5 to 4 if we bound it more carefully. 
\begin{lemma}\label{lemma:Pochhammer}
  Let $K\in \mathbb{N}$ and $K\ge 1$. We have $\prod_{j=1}^{\infty} \frac{1}{1-2^{-j/K}}\le 2^{5K}$.
\end{lemma}

\begin{proof}
  First, we bound the product of first $2K$ terms. Note that $\frac{1}{1-2^{-x}}\le \frac{8}{3x}$ for $0<x\le 2$. Therefore, 
  \begin{align}
    \prod_{j=1}^{2K}\frac{1}{1-2^{-j/K}}
    \le (8/3)^{2K}\cdot \frac{K^{2K}}{(2K)!}
    \le (8/3)^{2K}\cdot \frac{K^{2K}}{(2K/e)^{2K}}
    = (4e/3)^{2K}
    < 2^{4K}. 
  \end{align}
  
  Then, we bound the product of the rest terms
  \begin{align}
    \prod_{j=2K+1}^{\infty}\frac{1}{1-2^{-j/K}} 
    \le \prod_{j=2K+1}^{\infty}\frac{1}{1-2^{-\lfloor j/K \rfloor}} 
    \le \prod_{i=2}^{\infty}\left( \frac{1}{1-2^{-i}}\right)^K 
    \le \left( \frac{1}{1-\sum_{i=2}^{\infty}2^{-i}}\right)^K
    = 2^K.
  \end{align}
  
  Multiplying two parts proves the lemma.
\end{proof}

\begin{theorem}
  $\randcom^{\rightarrow,pub}_\delta(\ur^\subset) = \Omega(\log \frac{1}{\delta}\log^2 \frac{n}{\log (1/\delta)} )$, given that $64 \le \log \frac{1}{\delta} \le \frac{n}{64}$.
\end{theorem}

\begin{proof}
  By Lemma~\ref{lemma:zero-fail-prob}, the success probability of protocol $(\enc,\dec)$ is $1$. 
  By Lemma~\ref{lemma:lb-meta}, we have $\s\ge \log (^n_m) - \s' -1$, where $\s'=\log n + R+ \E(\log (^n_{|B|}))$. 
  The size of $B$ is $|B|=|S|-\sum_{r=1}^{R}{b_r}$.
  By Corollary~\ref{corollary:sampler-failure}, conditioned on $S$, $\Pr(b_r=0)\le \frac{I(X;A_{r-1})+1}{\log\frac{1}{\delta}}$. 
  By Lemma~\ref{lemma:mutual-entropy-bound}, $I(X;A_{r-1})\le 5K$ (Note that when $r=1$, $I(X;A_0)=0\le 5K$). Therefore, $\E(b_r)\ge 1-\frac{5K+1}{\log\frac{1}{\delta}}$.
  By the setting of parameters (see Algorithm~\ref{algo:para}) we have $\E(b_r)\ge \frac{5}{8}$. Therefore, $\E(|B|)\le |S|-\frac{5}{8}R$. 
  By Lemma~\ref{lemma:bits-saving}, $\log (^n_m)-\E(\log (^n_{|B|}))\ge \frac{5}{8}R\cdot \log (\frac{n}{m}-1) \ge \frac{1}{2}R\log (\frac{n}{\log(1/\delta)})$. 
  Furthermore, $\frac{1}{6}R\log \frac{n}{\log (1/\delta)} \ge R$.
  Combining together we get $\s \ge \frac{R}{3}\log \frac{n}{\log(1/\delta)} -(\log n + 1)  =\Omega(\log \frac{1}{\delta}\log^2 \frac{n}{\log (1/\delta)} )$.
\end{proof}
\section{Communication Lower Bound for $\ur_k^\subset$}\label{sec:k-samples-lb}

In this section, we prove a space lower bound of $\Omega(k\log^2
\frac{n}{k})$ bits for one-way communication complexity of $\ur_k^\subset$ (where $1\le k \le \frac{n}{2^{10}}$) with failure probability at most $\delta = \frac{1}{2}$.
Let $(\sketch_k, \query_k)$ denote a $\ur_k^\subset$-protocol. 
We consider the following encoding/decoding protocol $(\enc_k, \dec_k)$ for $S\in {[n] \choose m}$. 
$\enc_k$ will compute $M\leftarrow \sketch_k(\mathbf{1}_S)$ as part of its output. 
In addition, $\enc_k$ will output $B\subseteq S$ constructed as follows.
Initially $B\leftarrow S$, and $\enc_k$ proceeds in $R=\Theta(\log (n/k))$ rounds. 
Let $A_0=S\supseteq A_1\supseteq \ldots \supseteq A_R$ where $A_r$ is generated by sub-sampling each element in $A_{r-1}$ with probability $\frac{1}{2}$. 
On round $r$ ($r=1,\ldots, R$), $\enc_k$ tries to get $k$ elements from $A_{r-1}$ by invoking $\query_k(M, \mathbf{1}_{S\backslash A_{r-1}})$, denoted by $S_k$, and removes $S_k\cap (A_{r-1}\backslash A_{r})$ (whose expected size is $\frac{k}{2}$) from $B$. 
Note that $\dec_k$ is able to recover the elements in $S_k\cap (A_{r-1}\backslash A_{r})$. 
For each round the failure probability of $\query_k$ is at most $\delta$. 
Thus we have $\E(|S|-|B|)\ge \frac{k}{2}\cdot (1-\delta) \cdot R=\Omega(k\log\frac{n}{k})$. 
Furthermore, each element contains $\Theta(\log \frac{n}{k})$ bits of information, thus yielding a lower bound of $\Omega(k\log^2\frac{n}{k})$ bits.

\subsection{Protocol}
\begin{algorithm}[H] 
  \caption{Variables Shared by Encoder $\enc_k$ and Decoder $\dec_k$.} \label{algo:para4}
  \begin{algorithmic}[1] 
    \State $m\leftarrow \lfloor \sqrt{nk} \rfloor$
    \State $R\leftarrow \lfloor \frac{1}{2}\log (n/k) - 2 \rfloor$ \Comment{Note that $R\ge 3$ because $k\le \frac{n}{2^{10}}$}
    \State $T_0\leftarrow [n]$
    \For {$r = 1, \ldots, R$}
      \State $T_r\leftarrow \emptyset$
      \State For each $a\in T_{r-1}$, $T_r\leftarrow T_r\cup \{a\}$ with probability $\frac{1}{2}$ \Comment{We have $A_r=A\cap T_r$}
    \EndFor
  \end{algorithmic}
\end{algorithm}

\begin{algorithm}[H] 
  \caption{Encoder $\enc_k$.} \label{algo:enc4}
  \begin{algorithmic}[1]
    \Procedure{$\enc_k$}{$S$}
    \State $M \leftarrow \sketch_k(\mathbf{1}_S)$
    \State $D\leftarrow \emptyset$
    \For {$r=1,\ldots,R$}
    \State $S_r\leftarrow \query_k(M, \mathbf{1}_{S\backslash (S\cap T_{r-1})})$
    \If {$S_r\subseteq A\cap T_{r-1}$} \Comment{i.e. if $S_r$ are valid}
      \State $b_r\leftarrow 1$ \Comment{$b$ is a binary string of length $R$, indicating if $\query_k$ succeeds on round $r$}
      \State $D\leftarrow D \cup (S_r\cap (T_{r-1}\backslash T_r))$
    \Else 
      \State $b_r\leftarrow 0$
    \EndIf
    \EndFor
      \State \Return ($M$, $S\backslash D$, $b$) 
    \EndProcedure
  \end{algorithmic}
\end{algorithm}

\begin{algorithm}[H] 
  \caption{Decoder $\dec_k$.} \label{algo:dec4}
  \begin{algorithmic}[1]
    \Procedure{$\dec_k$}{$M$, $B$, $b$}
    \State $D\leftarrow \emptyset$
    \State $C_0 \leftarrow \emptyset$
    \For {$r=1,\ldots,R$}
      \State $C_r\leftarrow C_{r-1}$
      \If {$b_r=1$}
        \State $S_r\leftarrow \query_k(M, \mathbf{1}_{C_{r-1}})$ \Comment{Invariant: $C_r=A\backslash (A\cap T_r)$}
        \State $D\leftarrow D \cup (S_r\cap (T_{r-1}\backslash T_r))$
        \State $C_r\leftarrow C_r \cup (S_r\cap (T_{r-1}\backslash T_r))$
      \EndIf
      \State $C_r\leftarrow C_r \cup (B\cap (T_{r-1}\backslash T_r))$
    \EndFor
    \State \Return $B\cup D$ 
    \EndProcedure
  \end{algorithmic}
\end{algorithm}

\subsection{Analysis}

\begin{theorem}
  $\randcom^{\rightarrow,pub}_\delta(\ur_k^\subset) = \Omega(k\log^2 \frac{n}{k} )$, given that $1 \le k \le \frac{n}{2^{10}}$ and $\delta \le \frac{1}{2}$.
\end{theorem}

\begin{proof}
  Let $A_r=S\cap T_r$. 
  Let $\success$ denote the event that $|S\cap T_R|=|A_R|\ge k$. 
  Note that $\E(|A_R|)=\frac{1}{2^R}m=4k$. By Chernoff bound, $\Pr(\success)\ge \frac{1}{2}$. 
  In the following, we argue conditioned on $\success$. Namely, on each round $r$, there is at least $k$ items in $A_r$.  
  
  Similar to Lemma~\ref{lemma:zero-fail-prob}, we can prove the protocol $(\enc_k,\dec_k)$ always succeeds. 
  By Lemma~\ref{lemma:lb-meta}, we have $\s\ge \log (^n_m) - \s' -2$, where $\s'=\log n + R+ \E(\log (^n_{|B|}))$. 
  The size of $B$ is $|B|=|S|-\sum_{r=1}^{R}{(b_r \cdot |S_r \cap (A_{r-1}\backslash A_r)|)}$.
  Conditioned on $S$, the randomness used by $\ur_k^\subset$-protocol is independent from $S\backslash A_{r-1}$ (for $r=1, \ldots, R$).
  Therefore, $\E(b_r)\ge 1-\delta\ge \frac{1}{2}$, and $b_r$ is independent from $|S_r \cap (A_{r-1}\backslash A_r)|$. 
  We have $\E(|S_r \cap (A_{r-1}\backslash A_r)|)=\frac{k}{2}$, and thus $\E(|S|-|B|)\ge \frac{kR}{4}$. 
  By Lemma~\ref{lemma:bits-saving}, $\log (^n_m)-\E(\log (^n_{|B|}))\ge \frac{kR}{4}\cdot \log (\frac{n}{m}-1) \ge \frac{kR}{9}\log (\frac{n}{k})$.
  Moreover, $\frac{kR}{10}\log \frac{n}{k}\ge R$.  
  Combining together we get $\s = \Omega(kR\log\frac{n}{k}) = \Omega(k\log^2 \frac{n}{k} )$.
\end{proof}

\section{Lower bounds proofs via augmented indexing}\label{sec:aug-proof}

Here we show another route to proving $\randcom^{\rightarrow,pub}_\delta(\ur_k^\subset) = \Omega(\min\{n, t\log^2(n/t)\}$ via reduction from augmented indexing. We again separately prove lower bounds for $\randcom^{\rightarrow,pub}_\delta(\ur^\subset)$ and $\randcom^{\rightarrow,pub}_{\frac 15}(\ur_k^\subset)$. Both proofs make use of the following standard lemma. The proof can be found in the full version~\cite{KapralovNPWWY17}.

\begin{lemma}\label{lem:code}
For any integers $u\ge 1$ and $1\le m\le u/(4e)$, there exists a collection $\mathcal S_{u,m} \subset \binom{[u]}m$ with $\log |\mathcal{S}_{u,m}| = \Theta(m\log(u/m))$ such that for all $S\neq S'\in \mathcal S_{u,m}$, $|S\cap S'| < m/2$.
\end{lemma}

Both our lower bounds (for $\ur^\subset$ and $\ur_k^\subset$) reduce from augmented indexing (henceforth $\aug$) to either $\ur^\subset$ with low failure probability, or $\ur_k^\subset$ with constant failure probability, in the public coin one-way model of communication. We remind the reader of the setup for the $\aug_N$ problem. There are two players, Charlie and Diane. Charlie receives $z\in\{0,1\}^N$ and Diane receives $j^*\in[N]$ together with $z_{j^*+1},\ldots,z_N$. Charlie must send a single message to Diane such that Diane can then output $z_{j^*}$. The following theorem is known.

\begin{theorem}{\cite{MiltersenNSW98}}\label{thm:mnsw}
$\randcom^{\rightarrow,pub}_{1/3}(\aug_N) = \Theta(N)$.
\end{theorem}

We show that if there is an $s$-bit communication protocol $\mathcal P$ for $\ur^\subset$ on $n$-bit vectors with failure probability $\delta$ (or for $\ur_k$ with constant failure probability), that implies the existence of an $s$-bit protocol $\mathcal P'$ for $\aug_N$ for some $N=\Theta(\log\frac 1{\delta}\log^2\frac n{\log\frac 1{\delta}})$ (or $N=\Theta(k\log^2(n/k))$ for $\ur_k$). The lower bound on $s$ then follows from Theorem~\ref{thm:mnsw}.


%\subsection{Communication Lower Bound for $\ur^\subset$}\label{sec:aug-delta}

Set $t = \log \frac 1{\delta}$. In this section we assume $t < n/(4e)$ and show $\randcom^{\rightarrow,pub}_\delta(\ur^\subset) = \Omega(t\log^2(n/t))$. This implies a lower bound of $\Omega(\min\{n, t\log^2(n/t)\})$ for all $\delta>0$ bounded away from $1$.

As mentioned, we assume we have an $s$-bit protocol $\mathcal P$ for $\ur^\subset$ with failure probability $\delta$, with players Alice and Bob.We use $\mathcal P$ to give an $s$-bit protocol $\mathcal P'$ for $\aug_N$, which has players Charlie and Diane, for $N = \Theta(t\log^2(n/t))$.

\begin{algorithm}[H] 
  \caption{Behavior of Diane in $\mathcal P'$ for $\ur^\subset$.} \label{algo:diane1}
  \begin{algorithmic}[1]
    \Procedure{$\diane$}{$M$}
    \State $T \leftarrow \bigcup_{i=i^*+1}^L (\{i\} \times S_i \times [100^i])$
    \State $T_{i^*}\leftarrow \emptyset$
    \While {$|T_{i^*}| < \frac m2$}
      \State $(i,a,r)\leftarrow \pi^{-1}(\query(M, \mathbf{1}_{\pi(T)}))$
      \State $T\leftarrow T \cup ((i,a) \times [100^i])$
      \If {$i=i^*$} 
        \State $T_{i^*} \leftarrow T_{i^*}\cup \{a\}$
      \EndIf
    \EndWhile
    \If {there exists $S\in\mathcal S_{u_{i^*},m}$ with $T_{i^*}\subset S$}
      \State \Return the unique such $S$
    \Else
      \State \Return \textsf{Fail}
    \EndIf
    \EndProcedure
  \end{algorithmic}
\end{algorithm}


The protocol $\mathcal P'$ operates as follows. Without loss of generality we may assume that, using the notation of Lemma~\ref{lem:code}, $|\mathcal S_{u,m}|$ is a power of $2$ for $u, m$ as in the lemma statement. This is accomplished by simply rounding $|\mathcal S_{u,m}|$ down to the nearest power of $2$ by removing elements arbitrarily. Also, define $L = c\log(n/t)$ for some sufficiently small constant $c\in(0,1)$ to be determined later. Now, Charlie partitions the bits of his input $z\in\{0,1\}^N$ into $L$ consecutive sequences of bits such that the $i$th chunk of bits for each $i\in[L]$ can be viewed as specifying an element $S_i\in \mathcal S_{u_i,m}$ for $u_i = \frac n{100^i\cdot L}$ and $m = ct$. Lemma~\ref{lem:code} gives $\log|\mathcal S_{u_i,m}| = \Theta(m\log(u_i/m))$, which is $\Theta(t\log(n/t))$ for $c < 1/14$. Thus $N = \Theta(L\cdot t\log(n/t)) = \Theta(t\log^2(n/t))$. Given these sets $S_1,\ldots,S_L$, we now discuss how Charlie generates a vector $x\in\{0,1\}^n$. Charlie then simulates Alice on $x$ to generate the message Alice would have send to Bob in protocol $\mathcal P$, then sends that same message to Diane.

To generate $x\in\{0,1\}^n$, assume Charlie and Diane have sampled a bijection from 
$A = \bigcup_{i=1}^L (\{i\} \times [u_i]\times [100^i])$
to $[n]$ uniformly at random. We denote this bijection by $\pi$. This is possible since $|A| = n$. Then Charlie defines $x$ to be the indicator vector $\mathbf{1}_{\pi(S)}$, where
$S = \bigcup_{i=1}^L (\{i\} \times S_i \times [100^i])$,
then sends a message $M$ to Diane, equal to Alice's message with input $\mathbf{1}_{\pi(S)}$. This completes the description of Charlie's behavior in the protocol $\mathcal P'$.

We describe how Diane uses $M$ to solve $\aug_N$. Diane's input $j^*\in[N]$ lies in some chunk $i^*\in[L]$. We now show how Diane can use $\mathcal P$ to recover $S_{i^*}$ with probability $2/3$ (and thus in particular recover $z_{j^*}$). Since Diane knows $z_j$ for $j>j^*$, she knows $S_i$ for $i>i^*$. She can then execute the following algorithm.


In Algorithm~\ref{algo:diane1} Diane is building up a subset $T_{i^*}$ of $S_{i^*}$. Once $|T_{i^*}| \ge |S_{i^*}|/2 = m/2$, Diane can uniquely recover $S_{i^*}$ by the limited intersection property of $\mathcal{S}_{u_i,m}$ guaranteed by Lemma~\ref{lem:code}. Until then, she uses $\mathcal P$ to recover elements of $S\backslash T$, which, as we now show, are chosen uniformly at random from $S\setminus T$. 

The  proof of the following claim is deferred to the full version.
\begin{claim}\label{cl:uniform}
For every protocol for Alice and Bob that uses shared randomness with Bob's behaviour given by $\query(\cdot)$, for every choice of shared random string $R$ of Alice and Bob, for every $S, T\subset S$, the following conditions hold. If $\pi$ is a uniformly random permutation, the success or failure of $\query(M, \mathbf{1}_{\pi(T)})$ is determined by $\{\pi(j)\}_{j\in T}$ and the image $\pi(S\setminus T)$ of $S\setminus T$ under $\pi$. Conditioned on a choice of $R$, $\{\pi(j)\}_{j\in T}$ and $\pi(S\setminus T)$ such that $\query(M, \mathbf{1}_{\pi(T)})$ succeeds, one has that $\pi^{-1}(\query(M, \mathbf{1}_{\pi(T)}))$ is a uniformly random element of $S\setminus T$.
\end{claim}

\if 0 \begin{proof}
The first claim follows by noting that the message $M$ that Alice sends to Bob is solely a function of $R$ and $\pi(S)$. The behaviour of Bob is determined by $M$ and $\pi(T)$ (and the latter is determined by $\{\pi(j)\}_{j\in T}$).

Now condition on the values of $R$, $\{\pi(j)\}_{j\in T}$ and $\pi(S\setminus T)$ such that  $\query(M, \mathbf{1}_{\pi(T)})$ succeeds, and let $j^*\in [n]$ denote the output. Note that by our conditioning $j^*$ is a fixed quantity. The only randomness left is the exact mapping of $S\setminus T$ to $\pi(S\setminus T)$. This mapping is independent of $\{\pi(j)\}_{j\in T}$ and $\pi(S\setminus T)$ and uniformly random, so $\pi^{-1}(j^*)$ is a uniformly random element of $S\setminus T$, as required.
\end{proof}
\fi 
Fix any protocol $\widetilde{\query}(M, \mathbf{1}_{\pi(T)})$ (not necessarily the one that Charlie and Diane use; see analysis of the idealized process $\widetilde{\mathcal{P}}$ below). Now fix $T$ together with values of $R$, $\{\pi(j)\}_{j\in T}$ and $\pi(S\setminus T)$ such that  $\widetilde{\query}(M, \mathbf{1}_{\pi(T)})$ succeeds.  

\if 0
\paragraph{Elements in $S_{i^*}$ are likely to be recovered.}   Given Claim~\ref{cl:uniform}, since the elements of $S_i$ appear with frequency $100^i$ in $S\backslash T$, they are more likely to be returned by $\pi^{-1}(\widetilde{\query}(M, \mathbf{1}_{\pi(T)}))$. Indeed, as long as $T_{i^*}\le m/2$, at least $m/2$ elements remain in $S_{i^*}\backslash T_{i^*}$, implying the output of $(i, a, r)$ of $\pi^{-1}(\widetilde{\query}(M, \mathbf{1}_{\pi(T)}))$ satisfies
\begin{equation}
\begin{split}
\Pr(i = i^* | (R, \{\pi(j)\}_{j\in T}, \pi(S\setminus T)) \text{~s.t.~}\widetilde{\query}(M, \mathbf{1}_{\pi(T)})\text{~succeeds}) &\ge \frac{\frac m2\cdot 100^{i^*}}{\frac m2\cdot 100^{i^*} + m\cdot \sum_{i=1}^{i^*-1} 100^i}\\
& = \frac{\frac 12\cdot 100^{i^*}}{\frac 12\cdot 100^{i^*} + \frac {100}{99}(100^{i^*-1} - 1)}\\
& > \frac{49}{50} , \label{eqn:istar-likely}
\end{split}
\end{equation}
where the probability is over the choice of $\pi|_{S\setminus T}:(S\setminus T)\to \pi(S\setminus T)$ (recall that we condition on the image $\pi(S\setminus T)$ under $\pi$, but not on the actual mapping).
\fi


\textbf{Elements in $S_{j}, j<i^*,$ are unlikely to be recovered.} Given Claim~\ref{cl:uniform}, since the elements of $S_j$ appear with frequency $100^j$ in $S\backslash T$, they are less likely to be returned by $\pi^{-1}(\widetilde{\query}(M, \mathbf{1}_{\pi(T)}))$ for small $j$.  Specifically, as long as $|S_{i^*}\cap T_{i^*}|\geq m/2$, for any $j< i^*$
$\Pr(i = j | (R, \{\pi(j)\}_{j\in T}, \pi(S\setminus T)) \text{~s.t.~}\widetilde{\query}(M, \mathbf{1}_{\pi(T)})\text{~succeeds}) \le \frac{m\cdot 100^j}{\frac m2\cdot 100^{i^*}} \le 2\cdot 100^{-(i^*-j)} \le 50^{-(i^*-j)}$. %label{eqn:others-unlikely}
Here again the probability is over the choice of $\pi|_{S\setminus T}:(S\setminus T)\to \pi(S\setminus T)$ (recall that we condition on the image $\pi(S\setminus T)$ under $\pi$, but not on the actual mapping).

We now define the set $\mathcal{T}$ of {\bf typical intermediate sets}, which plays a crucial role in our analysis.  Let $Q_i$ for $i\in [L]$ denote $\{i\} \times S_i \times [100^i]$. Let $\mathcal{T}$ be the collection of all $T\subset S$ such that (1) $Q_i\subset T$ for all $i>i^*$, and (2) for each $i < i^*$, $|T\cap Q_i| \le 100^i\cdot  m/4^{i^*-i}$, such that if $T$ contains $\{(i, a, b)\}$ for some $a\in S_i$ and $b\in [100^i]$, then $T$ contains $\{(i, b)\times [100^i]\}$.  The following claim is proven in the full version:
\begin{claim}\label{cl:size-of-t}
The set $\mathcal{T}$ as above satisifies $|\mathcal{T}|=2^{O(m)}$.
\end{claim}
% \begin{proof}
% \allowdisplaybreaks
% \begin{align*}
% |\mathcal T| &\le 2^m \cdot \prod_{i=1}^{i^*-1}\left(\sum_{r=0}^{\frac m{4^{i^*-i}}} \binom mr\right) \\
% {}&\text{ (the }2^m\text{ term comes from }S_{i^*}\text{)}\\
% {}&\le 2^m \cdot \prod_{i=1}^{i^*-1} \binom{m + \frac m{4^{i^* - i}}}{\frac m{2^{i^* - i}}}\\
% {}&\le 2^m \cdot \prod_{i=1}^{i^*-1} (2e\cdot 4^{i^*-i})^{\frac m{4^{i^* - i}}}\text{ (using }\binom nk \le (en/k)^k\textrm{)}\\
% {}&\le 2^{O(m)} \cdot 2^{m\cdot O(\sum_{j=1}^\infty j 4^{-j})} \\
% {}& \le 2^{O(m)}
% \end{align*}
% \end{proof}


We will show that for most choices of $\pi$ and shared random string $R$ Algorithm~\ref{algo:diane1} {\bf (a)} never leaves the set $\mathcal{T}$ and {\bf (b)} successfully terminates.  Note that Algorithm~\ref{algo:diane1} is a random process whose sample space is the product of the set of all possible permutations $\pi$ and shared random strings $R$. As before, we denote this process by $\mathcal{P}'$. It is useful for analysis purposes to define another process $\widetilde{\mathcal{P}}$, which is an idealized version of $\mathcal{P}'$. In this process instead of running  $\query(M, \mathbf{1}_{\pi(T)})$ Alice runs  $\widetilde{\query}(M, \mathbf{1}_{\pi(T)})$, which is guaranteed to output an element of $\pi(S\setminus T)$ for every choice of  $T\subset S$, shared random string $R$, $\{\pi(j)\}_{j\in T}$, and $\pi(S\setminus T)$. The proof proceeds in three steps.



{\bf Step 1: proving that $\widetilde{\mathcal{P}}$ succeeds in recovering $T_{i^*}$ and never leaves $\mathcal{T}$ with high probability.}
Choose $\pi$ uniformly at random. By the upper bound on returning an element of $S_i$ above, as long as $|S_{i^*}\cap T_{i^*}|\geq m/2$, the expected number of items recovered by $\widetilde{\query}$ from $S_i$ for $i<i^*$ in the first $m$ iterations is at most $m/50^{i^*-i}$. Thus the probability of recovering more than $m/4^{i^*-i}$ items from $S_i$ is at most $(1/12)^{i^*-i}$ by Markov's inequality. Note that the probability is over the choice of $\pi$ only, as $\widetilde{\query}$ is assumed to succeed with probability $1$ by definition of $\widetilde{\mathcal{P}}$. 
Thus
$\Pr( \widetilde{\mathcal{P}}\text{~leaves~}\mathcal{T}) \le \sum_{i=1}^{i^*-1}\left(1/12\right)^{i^*-i} < 1/10.$
In particular this means that with probability at least $1-1/10$ at most $\sum_{i<i^*} m/4^{i^*-i}<m/2$ items from $\bigcup_{i<i^*} S_i$ are recovered in the first $m$ (or fewer, if the algorithm terminates earlier) iterations. This also implies that with probability at least $1-1/10$ if the algorithm proceeds for the entire $m$ iterations, it recovers at least $m/2$ elements of $T_{i^*}$ and hence terminates. We thus get that $\widetilde{\mathcal{P}}$ succeeds at least with probability $1-1/10$.

{\bf Step 2: coupling $\widetilde{\mathcal{P}}$ to $\mathcal{P}'$ on most of the probability space.}
For every $T\subset S$ and every $\pi$ let $\mathcal{E}_T(\pi)$ be the probabilistic event (over the choice of $\query$'s random string $R$) that $\query(M, \mathbf{1}_{\pi(T)})$ succeeds in returning an element in $\pi(S\backslash T)$. Note that $\mathcal{E}_T(\pi)$ is a subset of the probability space of shared random strings $R$, and depends on $\pi$. We let 
$\mathcal{E}_{\mathcal T}(\pi):=\wedge_{T\in\mathcal T} \mathcal E_T(\pi)$
to simplify notation. Using Claim~\ref{cl:size-of-t} and the union bound we have for every $\pi$
$\Pr_R(\neg(\mathcal E_{\mathcal T}(\pi))) \le \delta\cdot |\mathcal T|\leq 1/20$
as long as for $m = c\log(1/\delta)$ for $c$ a sufficiently small constant.

Now recall that $\widetilde{\query}(M, \mathbf{1}_{\pi(T)})$ is an idealized protocol, which is guaranteed to output an element of $\pi(S\setminus T)$ for every choice of  $T\subset S$, shared random string $R$, $\{\pi(j)\}_{j\in T}$, and $\pi(S\setminus T)$. We have just shown that for every $\pi$ the event ${\mathcal E_{\mathcal T}(\pi)}$ occurs  with probability at least $1-1/20$ over the choice of $R$. Now define $\widetilde{\query}(M, \mathbf{1}_{\pi(T)})$ as equal to $\query(M, \mathbf{1}_{\pi(T)})$ for all $T\in \mathcal{T}$ (the typical set of intermediate sets) and $(\pi, R)$ such that $R\in {\mathcal E_{\mathcal T}(\pi)}$, and extend $\widetilde{\query}(M, \mathbf{1}_{\pi(T)})$ to return an arbitrary element of $\pi(S\setminus T)$ for remaining tuples $(T, R, \pi(T),  \pi(S\setminus T))$. Note that $\widetilde{\query}$ defined in this way is a deterministic function once $T$, $R$, $\pi(T)$ and $\pi(S\setminus T)$ are fixed. 
Note that with probability at least $1-1/20$ over the choice of $\pi$ and $R$ one has $\query(M, \mathbf{1}_{\pi(T)})=\widetilde{\query}(M, \mathbf{1}_{\pi(T)})$ for all $T\in \mathcal{T}$, as required.


{\bf Step 3: arguing that $\mathcal{P}'$ succeeds with high probability.}  Choose $(\pi, R)$ uniformly at random. By {\bf Step 2} we have that with probability at least $1-1/20$ over this choice 
$\query(M, \mathbf{1}_{\pi(T)})=\widetilde{\query}(M, \mathbf{1}_{\pi(T)})$ for all $T\in \mathcal{T}$. At the same time we have by {\bf Step 1} that with probability at least $1-1/10$ over the choice of $\pi$ the idealized process $\widetilde{\mathcal{P}}$ 
succeeds in recovering $T_{i^*}$ and never leaves $\mathcal{T}$. Putting the two bounds together, we get that $\mathcal{P}'$ succeeds with probability at least $1-1/20-1/10>2/3$, showing the following theorem.


\begin{theorem}
For any $0<\delta<1/2$ and integer $n\ge 1$ with $\log \frac 1{\delta} < n/(4e)$, $\randcom^{\rightarrow,pub}_\delta(\ur^\subset) \ge \randcom^{\rightarrow,pub}_{1/3}(\aug_N)$ for $N = \Theta(\log\frac 1{\delta} \log^2 \frac n{\log \frac 1{\delta}})$.
\end{theorem}

\begin{corollary}
For any $0<\delta<1/2$ and integer $n\ge 1$, $\randcom^{\rightarrow,pub}_\delta(\ur^\subset) = \Omega(\min\{n, \log\frac 1{\delta} \log^2 \frac n{\log \frac 1{\delta}}\})$.
\end{corollary}


In the full version~\cite{KapralovNPWWY17} we use similar, but slightly simpler, ideas to lower bound $\randcom^{\rightarrow,pub}_{\frac 15}(\ur_k^\subset)$. 



% For peer review papers, you can put extra information on the cover
% page as needed:
% \ifCLASSOPTIONpeerreview
% \begin{center} \bfseries EDICS Category: 3-BBND \end{center}
% \fi
%
% For peerreview papers, this IEEEtran command inserts a page break and
% creates the second title. It will be ignored for other modes.
\IEEEpeerreviewmaketitle



% use section* for acknowledgement
\section*{Acknowledgment}
Initially the authors were focused on proving optimal lower bounds for samplers, but we thank Vasileios Nakos for pointing out that our $\ur^\subset$ lower bound immediately implies a tight lower bound for finding a duplicate in data streams as well. Also, initially our proof of Lemma~\ref{lem:information} incurred an additive $1$ in the numerator of the right hand side of \eqref{eqn:adaptivity}. This is clearly suboptimal for small $I(X; Y)$ (for example, consider $I(X; Y) = 0$, in which case the right hand side should be $\delta$ and not $1/\log(1/\delta)$)). We thank T.S.\ Jayram for pointing out that a slight modification of our proof could actually replace the additive $1$ with the binary entropy function (and also for showing us a different proof of this lemma, which resembles the standard proof of Fano's inequality).



% Can use something like this to put references on a page
% by themselves when using endfloat and the captionsoff option.
\ifCLASSOPTIONcaptionsoff
  \newpage
\fi


% trigger a \newpage just before the given reference
% number - used to balance the columns on the last page
% adjust value as needed - may need to be readjusted if
% the document is modified later
%\IEEEtriggeratref{8}
% The "triggered" command can be changed if desired:
%\IEEEtriggercmd{\enlargethispage{-5in}}


\bibliographystyle{alpha}
\bibliography{shortbib}

\newpage
\appendix

\section{Appendix}

\subsection{A tight upper bound for $\randcom^{\rightarrow,pub}_\delta(\ur_k)$}\label{sec:upper-bound}

In \cite[Proposition 1]{JowhariST11} it is shown that $\randcom^{\rightarrow,pub}_\delta(\ur_k) = O(\min\{n,t\log^2 n\})$ for $t = \max\{k,\log(1/\delta)\}$. Here we show that a minor modification of their protocol in fact shows the correct complexity $\randcom^{\rightarrow,pub}_\delta(\ur_k) = O(\min\{n,t\log^2(n/t)\})$, which given our new lower bound, is optimal up to a constant factor for the full range of $n,k,\delta$ as long as $\delta$ is bounded away from $1$.

Recall Alice and Bob receive $x, y\in\{0,1\}^n$, respectively, and share a public random string. Alice must send a single message $M$ to Bob, from which Bob must recover $\min\{k, \|x-y\|_0\}$ indices $i\in[n]$ for which $x_i\neq y_i$. Bob is allowed to fail with probability $\delta$. The fact that $\randcom^{\rightarrow,pub}_\delta(\ur_k) \le n$ is obvious: Alice can simply send the message $M = x$, and Bob can then succeed with failure probability $0$. We thus now show $\randcom^{\rightarrow,pub}_{e^{-ck}}(\ur_k) \le k\log^2(n/k)$ for some constant $c>0$, which completes the proof of the upper bound. We assume $k\le n/2$ (otherwise, Alice sends $x$ explicitly).

As mentioned, the protocol we describe is nearly identical to one in \cite{JowhariST11} (see also \cite{CormodeF14}). We will describe the new protocol here, then point out the two minor modifications that improve the $O(k\log^2 n)$ bound to $O(k\log^2(n/k))$ in Remark~\ref{rem:recov}. We first need the following lemma.

\begin{lemma}\label{lem:sparse-recov}
Let $\F_q$ be a finite field and $n>1$ an integer. Then for any $1\le k\le \frac n2$, there exists $\Pi_k\in \F_q^{m\times n}$ for $m = O(k\log_q(qn/k))$ s.t.\ for any $w\neq w'\in\F_q^n$ with $\|w\|_0, \|w'\|_0 \le k$, $\Pi_k w \neq \Pi_k w'$.
\end{lemma}
\begin{proof}
The proof is via the probabilistic method. $\Pi_k w = \Pi_k w'$ iff $\Pi_k (w - w') = 0$. Note $v = w-w'$ has $\|v\|_0 \le 2k$. Thus it suffices to show that such a $\Pi_k$ exists with no $(2k)$-sparse vector in its kernel. The number of vectors $v\in\F_q^n$ with $\|v_0\| \le 2k$ is at most $\binom{n}{2k}\cdot q^{2k}$. For any fixed $v$, $\Pr(\Pi_k v = 0) = q^{-m}$. Thus 
$$\Pr(\exists v, \|v\|_0 \le 2k: \Pi_k v = 0) \le \binom{n}{2k}\cdot q^{2k} \cdot q^{-m}$$ 
by a union bound. The above is strictly less than $1$ for $m > 2k + \log_q\binom{n}{2k}$, yielding the claim.
\end{proof}

\begin{corollary}\label{cor:ksparse}
Let $\F_q$ be a finite field and $n>1$ an integer. Then for any $1\le k\le \frac n2$, there exists $\Pi_k\in \F_q^{m\times n}$ for $m = O(k\log_q(qn/k))$ together with an algorithm $\mathcal{R}$ such that for any $w\in\F_q^n$ with $\|w\|_0 \le k$, $\mathcal{R}(\Pi_k w) = w$.
\end{corollary}
\begin{proof}
Given Lemma~\ref{lem:sparse-recov}, a simple such $\mathcal{R}$ is as follows. Given some $y = \Pi_k w^*$ with $\|w^*\|_0 \le k$, $\mathcal{R}$ loops over all $w$ in $\F_q^n$ with $\|w\|_0 \le k$ and outputs the first one it finds for which $\Pi_k w = y$.
\end{proof}

The protocol for $\ur_k$ is now as follows. Alice and Bob use public randomness to pick commonly known random functions $h_0,\ldots,h_L:[n]\rightarrow\{0,1\}$ for $L = \lfloor\log_2(n/k)\rfloor$, such that for any $i\in[n]$ and for any $j$, $\Pr(h_j(i) = 1) = 2^{-j}$. They also agree on a matrix $\Pi_{16k}$ and $\mathcal{R}$ as described in Corollary~\ref{cor:ksparse} for a sufficiently large constant $C>0$ to be determined later, with $q = 3$. Thus $\Pi_{16k}$ has $m = O(k\log(n/k))$ rows. Alice then computes $v_j = \Pi_{16k} x|_{h_j^{-1}(1)}$ for $j=0,\ldots,L$ where $v_j\in\F_q^m$, and her message to Bob is $M = (v_0,\ldots,v_L)$. For $S\subseteq [n]$ and $x$ an $n$-dimensional vector, $x|_S$ denotes the $n$-dimensional vector with $(x|_S)_i = x_i$ for $i\in S$, and $(x|_S)_i = 0$ for $i\notin S$. Note Alice's message $M$ is $O(k\log^2(n/k))$ bits, as desired. Bob then executes the following algorithm and outputs the returned values.

\begin{algorithm}[H] 
  \caption{Bob's algorithm in the $\ur_k$ protocol.} \label{algo:bob-protocol}
  \begin{algorithmic}[1]
    \Procedure{Bob}{$v_0,\ldots,v_L$}
    \For {$j=L,L-1,\ldots,0$}
      \State $v_j \leftarrow v_j - \Pi_{16k} y|_{h_j^{-1}(1)}$
      \State $w_j\leftarrow \mathcal{R}(v_j)$
      \If {$\|w_j\|_0 \ge k$ or $j=0$}
      \State \Return an arbitrary $\min\{k, \|w_j\|_0\}$ elements from $\supp(w_j)$
      \EndIf
    \EndFor
    \EndProcedure
  \end{algorithmic}
\end{algorithm}

The correctness analysis is then as follows, which is nearly the same as the $\ell_0$-sampler of \cite{JowhariST11}. If Alice's input is $x$ and Bob's is $y$, let $a = x-y \in \{-1,0,1\}^n$, so that $a$ can be viewed as an element of $\F_3^n$. Also let $a_j = a|_{h_j^{-1}(1)}$. Then $\E \|v_j\|_0 = \|a\|_0\cdot 2^{-j}$, and since $0\le \|a\|_0 \le n$, there either (1) exists a unique $0\le j^*\le L$ such that $2k\le \E\|a_j\|_0\cdot 2^{-j^*}< 4k$, or (2) $\|a\|_0 < 2k$ (in which case we define $j^* = 0$). Let $\mathcal{E}$ be the event that $\|a_j\|_0 \le 16k$ simultaneously for all $j\ge j^*$. Let $\mathcal{F}$ be the event that {\it either} we are in case (2), or we are in case (1) and $\|a_{j^*}\|_0 \ge k$ holds. Note that conditioned on $\mathcal{E}, \mathcal{F}$ both occurring, Bob succeeds by Corollary~\ref{cor:ksparse}.

We now just need to show $\Pr(\neg\mathcal{E} \wedge \neg\mathcal{F}) < e^{-\Omega(k)}$. We use the union bound. First, consider $\mathcal{F}$. If $j^* = 0$, then $\Pr(\neg\mathcal{F}) = 0$. If $j^*\neq 0$, then $\Pr(\neg\mathcal{F}) \le \Pr(\|a_{j^*}\|_0 < \frac 12 \cdot\E\|a_{j^*}\|_0)$, which is $e^{-\Omega(k)}$ by the Chernoff bound since $\E\|a_{j^*}\|_0 = \Theta(k)$. Next we bound $\Pr(\neg \mathcal{E})$. For $j\ge j^*$, we know $\E\|a_j\|_0 \le 4k/2^{j-j^*}$. Thus, letting $\mu$ denote $\E\|a_j\|_0$, 
\begin{equation}
\Pr(\|a_j\|_0 > 16k) < \left(\frac{e^{\frac{16k}{\mu} - 1}}{(\frac{16k}{\mu})^{\frac{16k}{\mu}}}\right)^\mu < \left(\frac{16k}{\mu}\right)^{-\Omega(k)} < (e^{-Ck})^{j-j^*}\label{eqn:geometric}
\end{equation}
for some constant $C>0$ by the Chernoff bound and the fact that $16k/\mu \ge 4 > e$. Recall that the Chernoff bound states that for $X$ a sum of independent Bernoullis,
$$
\forall \delta > 0,\ \Pr(X > (1+\delta) \E X) < \left(\frac{e^\delta}{(1+\delta)^{1+\delta}}\right)^{\E X} .
$$
Then by a union bound over $j\ge j^*$ and applying \eqref{eqn:geometric},
$$
\Pr(\neg \mathcal{E}) = \Pr(\exists j\ge j^*: \|a_j\|_0 > 16k) < \sum_{j=j^*}^\infty (e^{-Ck})^{j-j^*} = O(e^{-Ck}) .
$$

\begin{remark}\label{rem:recov}
\textup{
As already mentioned, the protocol given above and the one described in \cite{JowhariST11} using $O(k\log^2 n)$ bits differ in minor points. First: the protocol there used $\lfloor\log_2 n\rfloor$ different hash functions $h_j$, but as seen above, only $\lfloor \log_2(n/k)\rfloor$ are needed. This already improves one $\log n$ factor to $\log(n/k)$. The other improvement comes from replacing the $k$-sparse recovery structure with $2k$ rows used in \cite{JowhariST11} with our Corollary~\ref{cor:ksparse}. Note the matrix $\Pi_k$ in our corollary has even {\it more} rows, but the key point is that the bit complexity is improved. Whereas using a $k$-sparse recovery scheme as described in \cite{JowhariST11} would use $2k$ linear measurements of a $k$-sparse vector $w\in\{-1,0,1\}^n$ with $\log n$ bits per measurement (for a total of $O(k\log n)$ bits), we use $O(k\log(n/k))$ measurements with only $O(1)$ bits per measurement. The key insight is that we can work over $\F_3^n$ instead of $\R^n$ when the entries of $w$ are in $\{-1,0,1\}$, which leads to our slight improvement.
}
\end{remark}

\subsection{Proof of the existence of the desired $\mathcal S_{u,m}$}\label{sec:code}
\noindent \textbf{Lemma~\ref{lem:code} (restated).}
For any integers $u\ge 1$ and $1\le m\le u/(4e)$, there exists a collection $\mathcal S_{u,m} \subset \binom{[u]}m$ with $\log |\mathcal{S}_{u,m}| = \Theta(m\log(u/m))$ such that for all $S\neq S'\in \mathcal S_{u,m}$, $|S\cap S'| < m/2$.
\begin{proof}
The proof is via the probabilistic method. We pick $S_1,\ldots,S_N$ independently, each one uniformly at random from $\binom{[u]}m$. Fix $i\neq j\in[N]$. Imagine $S_i$ being fixed and picking the $m$ elements of $S_j$ one by one. Let $X_k$ denote the indicator random variable for the event that the $k$th element picked is also in $S_i$. Then $|S_i\cap S_j| = \sum_{k=1}^m X_k$, and we set $\mu:= \E |S_i\cap S_j|$, which is $m^2/u$ by linearity of expectation. We have $\Pr(|S_i \cap S_j| \ge m/2) = \Pr(|S_i\cap S_j| \ge (1+\delta)\mu)$ for $\delta = u/(2m) - 1$. The $X_k$ are not independent, but they are negatively dependent. Thus the Chernoff bound yields
$$
\Pr(|S_i\cap S_j| \ge (1+\delta)\mu) \le \left(\frac{e^{\delta}}{(1+\delta)^{1+\delta}}\right)^\mu \le \left(\frac{e^{\frac u{2m} - 1}}{(\frac u{2m})^{\frac u{2m}}}\right)^{m^2/u} \le \left(\frac u{2em}\right)^{-\frac m2} .
$$
Setting $N = \sqrt{(u/(2em))^{m/2} - 1}$ so that ${N \choose 2}\leq N^2=(u/(2em))^{m/2} - 1$, by a union bound with positive probability $|S_i\cap S_j| < m/2$ for all $i\neq j$, simultaneously, as desired. Note for this choice of $N$, we have $\log|\mathcal S_{u,m}| = \log N = \Theta(m\log(u/m))$.
\end{proof}

\subsection{Another variant of samplers, with applications}\label{sec:variant}

In this section we define another variant of the $\ell_0$-sampling problem and provide a solution for it. The solution is a minor modification of the $\ell_0$-sampling algorithm of \cite{JowhariST11}.

\begin{definition}
In the turnstile streaming problem {\em $\ell_0$-sampling$_k$($\delta_1,\delta_2$)}, there is a vector $z\in\R^n$ receiving turnstile streaming updates, and the answer to \texttt{query}() must behave as follows:
\begin{itemize}
\item With probability at most $\delta_1$, the output can be ``\textsf{Fail}''.
\item With probability at most $\delta_2$, the output can be arbitrary.
\item Otherwise, the output should be a uniformly random subset of size $\min\{k,\|z\|_0\}$, without replacement, from $\supp(z)$.
\end{itemize}
\end{definition}

One can define the \suppfind{k}$(\delta_1,\delta_2)$ problem analogously, as well as similar variants for other desired output distributions. The important distinction is that there are two types of failures, which are allowed to happen with different probabilities.

The description of algorithm given in \cite{AhnGM12a} for outputting connected components is based on a \suppfind{} subroutine that always knows when it fails and outputs \textsf{Fail} at those times, and \cite{AhnGM12a} cites \cite{JowhariST11} for implementing this subroutine. Unfortunately, the algorithm of \cite{JowhariST11} does not provide this behavior and can behave arbitrarily when it fails. One can avoid this issue by simply conditioning on never failing, which would require \cite{AhnGM12a} to call the \cite{JowhariST11} subroutine with $\delta < 1/\mathop{poly}(n)$. This would worsen their claimed space for finding connected components from $O(n\log^3 n)$ bits to $O(n\log^4 n)$ bits. As mentioned in Footnote~\ref{notejst} though, it is apparent from the correctness analysis of the algorithm in \cite{AhnGM12a} that their algorithm actually only needs a \suppfind{1}$(\delta_1,\delta_2)$ data structure for $\delta_1$ a small but universal constant, and $\delta_2 = 1/\mathop{poly}(n)$. As we sketch below, $\ell_0$-sampling$_k(\delta_1,\delta_2)$ can be solved in the general turnstile model in $O((t\log n + \log(n/\delta_2)) \log (n/t))$ bits of space for $t = \max\{k, \log(1/\delta_1)\}$, based on modifying the algorithm of \cite{JowhariST11}. This leads to an implementation of the connectivity algorithm of \cite{AhnGM12a} using space $O(n\log^3 n)$ bits as they claim.

We will need the following standard lemma.

\begin{lemma}\label{lem:zero-test}
There is a turnstile streaming algorithm for testing whether a vector $z\in\R^n$ updated in a stream is identically zero. If $z=0$, the algorithm outputs $z=0$. If $z\neq 0$, the algorithm outputs $z=0$ with probability at most $\delta$. The space consumption is $O(\log(nT/\delta))$ bits, assuming that the entries of $z$ are always integers bounded by $T$ in magnitude.
\end{lemma}
\begin{proof}
We pick a prime $p$ larger than $T + n/\delta$. We then pick a random $r\in\mathbb{F}_p$ at the beginning of the stream, and throughout the stream we maintain a single number in memory: $q(r) = \sum_{i=1}^n z_i r^i\mod p$. That is, during an update ``$z_i\leftarrow z_i + \Delta$'', we add $\Delta \cdot r^i$ to our counter, where the addition, multiplication, and exponentiation are all in $\mathbb{F}_p$. Note if $z\neq 0$, $q$ is a non-zero degree-$n$ polynomial (since $p > T$) and thus has at most $n$ zeroes in $\mathbb{F}_p$, implying that $\Pr_r(q(r) = 0) \le n/p < \delta$.
\end{proof}

We now sketch the modification to the algorithm of \cite{JowhariST11} which solves the above variant of $\ell_0$-sampling with the desired space complexity. We assume that the vector $z$ always has integer entries and $\|z\|_\infty \le T = \mathop{poly}(n)$ throughout the stream. First we describe the algorithm without regard for the space needed to store hash functions. For fixed $k$ and $\delta_2$, we give an algorithm which achieves failure probability $\delta_1 = \exp(-\Theta(k))$ so that $t = \Theta(k)$. The algorithm of \cite{JowhariST11} is then similar to the upper bound for $\ur$ in Section~\ref{sec:upper-bound} (see also \cite{CormodeF14}). Here we set $L = \Theta(\log(n/k))$. We then at each level, as in Section~\ref{sec:upper-bound}, use a $Ck$-sparse recovery scheme for vectors in $\mathbb{F}_p^n$ for some prime $p > \max\{n,T\}$ and constant $C>0$ to arise in the analysis later (so that taking each entry of the recovered vector modulo $p$ still returns the same vector). Such a measurement matrix $\Pi_{Ck}$ only needs $2Ck$ measurements over $\mathbb{F}_p$ and thus the recovery scheme uses $O(k\log p) = O(k\log n)$ bits. For example, one can use $\Pi$ being a $2Ck\times n$ Vandermonde matrix, so that no $Ck$-sparse vector is in its kernel. Recovery can be done by brute force. Alternatively one can use syndrome decoding, which uses the same space but allows recovery in time $\mathop{poly}(k\log p)$ time (see \cite[Section E]{DodisORS08}). Now, suppose level $j^*$ is the level we return our output from, as per the scheme in Algorithm~\ref{algo:bob-protocol}. If $w_j$ denotes the output of the recovery scheme, before returning $w_j$ we first would like to check that $w_j = a_j$ with $a_j$ denoting $z|_{h_j^{-1}(1)}$, i.e. that $d_j = w_j - a_j$ is equal to zero. For this we use the zero-tester algorithm of Lemma~\ref{lem:zero-test} as a subroutine, with failure probability parameter $\delta_2$. Thus overall we obtain the desired guarantee with space $O((k\log n + \log(n/\delta_2))\log(n/k))$, as desired. The analysis of correctness is the same as in Section~\ref{sec:upper-bound}, based on the same events $\mathcal E, \mathcal F$. The difference is that $\mathcal E$ or $\mathcal F$ fail to hold, which happens with probability at most $\delta_1$, then if the level $j$ we base our output on does not provide us with a $k$-sparse vector, then we catch this with the zero-tester and say \textsf{Fail} (except with probability $\delta_2$).

Now we must take into account the space to store the hash functions $h_0,\ldots,h_L$. We implement these hash functions by a single hash function $h:[n]\rightarrow[n]$ where we then interpret the event ``$h_j(i) = 1$'' as occurring when the least significant bit of $h(i)$ is equal to $j$. We now just need to derandomized the selection of this single hash function. For this, we need to make sure that $h$ is still chosen in suc ha way that $\Pr_h(\neg\mathcal E \vee \neg\mathcal F) < \delta_1$. For the purposes of this algorithm, we modify the definition of $\mathcal E$ to be stronger: here we consider it to be the event that $\sum_{j=j^*}^L \|a_j\|_0 \le Ck$ for some constant $C>0$, i.e.\ the {\em total} number of elements at levels $j^*$ and deeper is $O(k)$. Then $\Pr(\neg \mathcal E) < \exp(-\Omega(k))$ is still true by a union bound and Chernoff bound as in Section~\ref{sec:upper-bound}, and $\mathcal F$ is not modified from that section, so we also have $\Pr(\neg\mathcal F) < \exp(-\Omega(k))$, as desired. Now, we cannot afford to store a truly random $h:[n]\rightarrow[n]$, so we instead determine $h$ using Nisan's pseudorandom generator (PRG) \cite{Nisan92}. More specifically, it suffices to fool the following two branching programs up to statistical distance $\delta_2$, one branching program for each of $\mathcal E, \mathcal F$. For $\mathcal E$, we have a branching program that sees the values $h(1),\ldots,h(n)$ in order and maintains a single word of memory which simply counts the number of items $i\in[n]$ that then hash to a level $j\ge j^*$. The space of this branching program is then $\log n$ bits which is certainly at most $S = \log(n/\delta_2)$, and the randomness needed is $R = n\log n$. Thus when using Nisan's PRG, the probability of $\mathcal E$ occurring changes by at most an additive $\exp(-\Omega(S)) < \delta_2$, and the required seed length for Nisan's PRG is $O(S\log(R/S)) = O(\log(n/\delta_2)\log n)$ bits. For the next branching program, recall that $\mathcal F$ is the event that $\|a_{j^*}\|_0 \ge k$. We thus set up a similar branching program, which sees $h(1),\ldots,h(n)$ in order and counts the number of $i\in[n]$ hashing to level $j^*$. The seed length and error are the same as for the first branching program. We have thus established the following theorem.

\begin{theorem}
For any $k\ge 1$ and $0<\delta_1,\delta_2<1$, there is a solution to the $\ell_0$-sampling$_k$($\delta_1,\delta_2$) in turnstile streams with space complexity $O((t\log n + \log(n/\delta_2))\log(n/t))$ bits, where $t = \max\{k, \log(1/\delta_1)\}$.
\end{theorem}




% that's all folks
\end{document}


