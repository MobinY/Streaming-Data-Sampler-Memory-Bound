\begin{abstract}
%\boldmath
In the communication problem $\ur$ (universal relation) \cite{KarchmerRW95}, Alice and Bob respectively receive $x, y \in\{0,1\}^n$ with the promise that $x\neq y$. The last player to receive a message must output an index $i$ such that $x_i\neq y_i$. We prove that the randomized one-way communication complexity of this problem in the public coin model is exactly $\Theta(\min\{n,\log(1/\delta)\log^2(\frac n{\log(1/\delta)})\})$ for failure probability $\delta$. Our lower bound holds even if promised $\mathop{support}(y)\subset \mathop{support}(x)$. As a corollary, we obtain optimal lower bounds for $\ell_p$-sampling in strict turnstile streams for $0\le p < 2$, as well as for the problem of finding duplicates in a stream. Our lower bounds do not need to use large weights, and hold even if promised $x\in\{0,1\}^n$ at all points in the stream. 

We give two different proofs of our main result. The first proof demonstrates that any algorithm $\mathcal A$ solving sampling problems in turnstile streams in low memory can be used to encode subsets of $[n]$ of certain sizes into a number of bits below the information theoretic minimum. Our encoder makes adaptive queries to $\mathcal A$ throughout its execution, but done carefully so as to not violate correctness. This is accomplished by injecting random noise into the encoder's interactions with $\mathcal A$, which is loosely motivated by techniques in differential privacy. Our correctness analysis involves understanding the ability of $\mathcal A$ to correctly answer adaptive queries which have positive but bounded mutual information with $\mathcal A$'s internal randomness, and may be of independent interest in the newly emerging area of adaptive data analysis with a theoretical computer science lens. Our second proof is via a novel randomized reduction from Augmented Indexing \cite{MiltersenNSW98} which needs to interact with $\mathcal A$ adaptively. To handle the adaptivity we identify certain likely interaction patterns and union bound over them to guarantee correct interaction on all of them. To guarantee correctness, it is important that the interaction hides some of its randomness from $\mathcal A$ in the reduction.
\end{abstract}
% IEEEtran.cls defaults to using nonbold math in the Abstract.
% This preserves the distinction between vectors and scalars. However,
% if the journal you are submitting to favors bold math in the abstract,
% then you can use LaTeX's standard command \boldmath at the very start
% of the abstract to achieve this. Many IEEE journals frown on math
% in the abstract anyway.

% Note that keywords are not normally used for peerreview papers.
\begin{IEEEkeywords}
IEEEtran, journal, \LaTeX, paper, template.
\end{IEEEkeywords}
