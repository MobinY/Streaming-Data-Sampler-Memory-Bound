\section{$\Omega(\log^2 n \log \frac{1}{\delta})$ Bits Lower Bound} \label{sec:optimal-lb}

In the previous section we have shown how to extract $\Theta(\log n\log \frac{1}{\delta})$ bits of information from a sampler. 
Our goal now is to extract more words. Our improved lower bound is motivated by the best known $\ell_0$-sampler {\em upper} bound \cite{JowhariST11}, which uses $O(\log^2 n\log \frac 1{\delta})$ bits of memory. At a high level, this upper bound works as follows. The sampler consists of $\log n$ layers, and on layer $i=1,\ldots, \log n$ it maintains a separate $\log \frac 1{\delta}$-sparse recovery system for the sub-stream generated by sub-sampling the items from the universe $[n]$ with probability $2^{-i}$. 
Each sparse recovery system takes $O(\log n\log \frac 1{\delta})$ bits.  Its correctness comes from the fact that 
\begin{enumerate}
  \item With probability at least $1-n^{-c}$ there is some layer $i$ that contains at least $1$ and at most $\log n$ items whose coordinates are non-zero. 
  \item Conditioned on the event that on layer $i$ the number of items whose coordinates are non-zero is between $1$ and $\log n$, the sparse recovery system on layer $i$ works with failure probability at most $n^{-c}$. In this context, we say the sparse recovery system works if it could recover at least one item. 
\end{enumerate}

Intuitively, the previous lower bound only extracts information from one single layer (i.e. layer $0$). In this section, we build a framework to extract information from multiple layers. A second technique that makes our improvement possible is random removal: after obtaining a sample from the sampler, we not only peel off the sample, but also randomly remove a fraction of items remaining in the sampler. 

\subsection{Protocol}

The parameters used by Alice and Bob are given in Algorithm~\ref{algo:para}.
Alice wants to send a uniformly random set $A$ to Bob where $|A|=m$. Similar to $\enc_1$, Alice constructs \samp by inserting all the items in $A$, and sends it to Bob. Note that \samp uses the same random source that Alice and Bob share. Moreover, Alice will send Bob a subset $B\subseteq A$ computed as follows. Initially $B=A$ and let $A_0=A$. Alice proceeds in $R$ rounds. On round $r$ ($r=1,\ldots, R$) Alice tries to obtain sample $s_r\in A_{r-1}$ from the sampler 
(In more detail, she will make a copy of \samp, denoted by $\samp_r$, remove all the items in $A\backslash A_{r-1}$ and query for a sample). 
Let $b$ denote a binary string of length $R$, where $b_r$ records whether the sampler succeeds on round $r$. In the end, Alice will also send $b$ to Bob. If $s_r\in A_{r-1}$, i.e., the sampler returns a valid sample, Alice will set $b_r=1$, and remove $s_r$ from $B$ (since $s_r$ can be obtained from the sampler, Alice does not need to include it in $B$. In Algorithm~\ref{algo:enc} Alice uses $S$ to keep track of $A\backslash B$); otherwise Alice will set $b_r=0$.
At the end of round $r$, Alice will generate a uniformly random set $A_r$, so that $A_r$ is a subset of $A_{r-1}\backslash \{s_r\}$ and $|A_r|=n_r$. 
In particular, Alice uses shared random source to generate $A_r$, so that Bob can recover $A_{r-1}\backslash A_r$ on round $r$. 
We present Alice's encoder in Algorithm~\ref{algo:enc}.

The decoding process is symmetric. Let $C_0=\emptyset$ and $S=\emptyset$. Bob proceeds in $R$ rounds. 
On round $r$ ($r=1,\ldots,R$), Bob obtains sample $s_r\in A\backslash C_{r-1}$ (In more detail, he will make a copy of \samp, denoted by $\samp_r$, remove all the items in $C_{r-1}$ and query for a sample). 
By the construction of $C_{r-1}$ that will be described later it is guaranteed that $A_{r-1}=A\backslash C_{r-1}$. 
Therefore, Bob will get exactly the same $s_r$ as Alice. 
Bob assigns initial value $C_{r-1}$ to $C_r$.
If $b_r=1$, Bob will add $s_r$ to both $S$ and $C_r$.
At the end of round $r$, Bob inserts a bunch of items to $C_r$ so that $C_r=A\backslash A_r$. Bob can achieve this because of the shared randomness when constructing $A_r$.
In the end, Bob's decoder outputs $B\cup S$.
We present the decoder in Algorithm~\ref{algo:dec}.

\begin{algorithm}[H] 
  \caption{Variables Shared by Alice's $\enc$ and Bob's $\dec$.} \label{algo:para}
  \begin{algorithmic}[1] 
    \State $m\leftarrow n^{0.99}$
    \State $K\leftarrow \frac{1}{10}\log \frac{1}{\delta}$
    \State $R\leftarrow \frac{1}{200}\log n \log \frac{1}{\delta}$
    \For {$r = 0, \ldots, R$}
      \State $n_r\leftarrow m \cdot 2^{-\frac{r}{K}}$ \Comment{By the setting of parameters we have $n_r-n_{r+1}\ge 2$}
    \EndFor
    \For {$a\in [n]$}
      \State Let $U_a$ be uniformly and independent sample from $[0,1]$ \Comment{Used to generate $A_r$, $r=1,\ldots, R$}
    \EndFor
  \end{algorithmic}
\end{algorithm}

\begin{algorithm}[H] 
  \caption{Alice's Encoder.} \label{algo:enc}
  \begin{algorithmic}[1]
    \Procedure{$\enc$}{$A$}
    \State Initialize an $\ell_0$-sampler $\samp$
    \State Insert items in $A$ into \samp
    \State $A_0 \leftarrow A$
    \State $S\leftarrow \emptyset$
    \For {$r=1,\ldots,R$}
      \State Let $\samp_r$ be a copy of $\samp$
      \State Remove items in $A\backslash A_{r-1}$ from $\samp_r$ \Comment So that $\samp_r$ contains the elements in $A_{r-1}$
      \State Let $s_r$ be a sample from $\samp_r$
      \State $A_r\leftarrow A_{r-1}$
      \If {$s_r\in A_r$} \Comment{i.e. if $s_r$ is a valid sample}
        \State $b_r\leftarrow 1$ \Comment{$b$ is a binary string of length $R$, indicating if sampler succeeds on round $r$}
        \State $S\leftarrow S \cup \{s_r\}$
        \State $A_r\leftarrow A_r \backslash \{s_r\}$
      \Else 
        \State $b_r\leftarrow 0$
      \EndIf
      \State Remove $|A_r|-n_r$ elements from $A_r$ with smallest $U_a$'s among $a\in A_r$ \Comment{So that $|A_r|=n_r$}
    \EndFor
    \State \Return ($A\backslash S$, $b$, \samp) 
    \EndProcedure
  \end{algorithmic}
\end{algorithm}

\begin{algorithm}[H] 
  \caption{Bob's Decoder.} \label{algo:dec}
  \begin{algorithmic}[1]
    \Procedure{$\dec$}{$B$, $b$, \samp}
    \State $S\leftarrow \emptyset$
    \State $C_0 \leftarrow \emptyset$
    \For {$r=1,\ldots,R$}
      \State $C_r\leftarrow C_{r-1}$
      \If{$b_r=1$}
        \State Let $\samp_r$ be a copy of $\samp$
        \State Remove items in $C_{r-1}$ from $\samp_r$ \Comment{Invariant: $C_r=A\backslash A_r$ ($A_r$ is defined in $\enc$)}
        \State Let $s_r$ be a sample from $\samp_r$
        \State $S\leftarrow S \cup \{s_r\}$
        \State $C_r\leftarrow C_r \cup \{s_r\}$
      \EndIf
       \State Insert $m-n_r-|C_r|$ items into $C_r$ with smallest $U_a$'s among $a\in B\backslash C_r$
    \EndFor
    \State \Return $B\cup S$ 
    \EndProcedure
  \end{algorithmic}
\end{algorithm}

\subsection{Analysis}

We have three random objects here: (1) Set $A$, which is a uniform random set of size $m$; (2) The random source used by the sampler, denoted by $X$; (3) The collection of random variables $U_a$ where $a\in [n]$, denoted by $U$. They are independent. We do the analysis conditioned on $A$. Namely, the following arguments apply for any fixed $A$. 

First, we can prove that $\dec(\enc(A))=A$. That is, no matter the randomness in $X$ and $U$, Bob will always decode $A$ successfully. It is because Alice's $\enc$ and Bob's $\dec$ share $X$ and $U$, so that Bob essentially simulates Alice. We formally prove this by induction in Lemma~\ref{lemma:zero-fail-prob}. 

Now our goal is to prove that by using \samp, the number of bits that Alice saves is $\Omega(\log^2 n \log\frac{1}{\delta})$. Intuitively, it is equivalent to prove the number of words that Alice saves is $\Omega(\log n \log\frac{1}{\delta})$.
We formalize this in Lemma~\ref{lemma:bits-saving}. Note that Alice also needs to send $b$ (i.e., whether the sampler succeeds on $R$ rounds) to Bob, but it takes only $O(\log n \log\frac{1}{\delta})$ bits, which is negligible. Thus it is sufficient to prove $\E(|B|)=|A|-\Omega(\log n \log\frac{1}{\delta})$. 

We have $|A|-|B|=\sum_{r=1}^{R}b_r$. 
In Lemma~\ref{lemma:mutual-entropy-vs-fail-prob}, we prove the probability that the sampler fails on round $r$ is upper bounded by $\frac{I(X;A_{r-1})+1}{\log \frac{1}{\delta}}$, where $I(X;A_{r-1})$ is the mutual information between $X$ and $A_{r-1}$. 
Furthermore, we will show in Lemma~\ref{lemma:mutual-entropy-bound} that $I(X;A_{r-1})$ is upper bounded by $5K$.
Combining together we get $\E(b_r)\ge 1-\frac{5K+1}{\log \frac{1}{\delta}}\ge \frac{2}{5}$ and thus $\E(|A|-|B|)\ge \frac{2}{5}R=\Omega(\log n \log\frac{1}{\delta})$. 

\begin{lemma}\label{lemma:zero-fail-prob}
  $\dec(\enc(A))=A$.
\end{lemma}

\begin{proof}
  We claim that for $r=0,\ldots, R$, $\{A_r, C_r\}$ is a partition of $A$ ($A_r$ is in \enc, and $C_r$ is in \dec). We prove the claim by induction on $r$.
  
  The basis case is $r=0$. $A_0=A$ and $C_0=\emptyset$. The claim is true.
  
  Assume by induction the claim holds for $r$ ($0\le r < R$), and we consider $r+1$. 
  On round $r$, $\samp_r$ is \samp with items in $A\backslash A_{r-1}=C_{r-1}$ removed in both \enc and \dec. Therefore, the sample $s_r$ obtained from $\samp_r$ in both sides is the same. Initially $A_r=A_{r-1}$ and $C_r=C_{r-1}$, and so $\{A_r,C_r\}$ is a partition of $A$. If $s_r$ is a valid sample (i.e. $s_r\in A_{r-1}$), then $b_r=1$, and \enc removes $s_r$ from $A_r$ and in the meanwhile \dec inserts $s_r$ into $C_r$, so that $\{A_r, C_r\}$ remains a partition of $A$. 
  
  After that, Alice's \enc repeats removing $a$ from $A_r$ with smallest $U_a$ until $|A_r|=n_r$. Symmetrically, Bob's \dec repeats inserting $a$ into $C_r$ with smallest $U_a$ among $a\in B\backslash C_r$, until $|C_r|=|A|-n_r$. In the end we have $|A_r|+|C_r|=|A|$, so \enc and \dec execute repetition the same number of times. Moreover, we can prove that during the same repetition the item removed from $A_r$ is exactly the same item inserted to $C_r$. This is because in the beginning of a repetition $\{A_r, C_r\}$ is a partition of $A$. We have $B\backslash C_r\subseteq A\backslash C_r=A_r$. Let $a^*$ denote  $a\in A_r$ that minimizes $U_a$. We can prove that $a^*\in B\backslash C_r\subseteq A_r$ (since $a^*$ will be removed from $A_r$, it has no chance to be included in $S$ in \enc, so that $B$ contains $a^*$), and $U_{a^*}$ is also the smallest among $\{U_a|a\in B\backslash C_r\}$. Thus in the end of the repetition, both Alice and Bob will take $a^{*}$ to update (remove from $A_r$, insert into $C_r$). Therefore, $\{A_r, C_r\}$ remains a partition of $A$.
  
  Given the fact that $\{A_r, C_r\}$ is a partition of $A$, $s_r$ is the same in \enc and \dec. Furthermore, $S=\{s_r|b_r=1,r=1,\ldots, R\}$ is the same in \enc and \dec. We know $S\subseteq A$. Since \enc outputs $A\backslash S$, and \dec outputs $(A\backslash S)\cup S$, we have $\dec(\enc(A))=A$. 
\end{proof}

\begin{lemma} \label{lemma:bits-saving}
  Let $m=n^{0.99}$. Let $W\in \mathbb{N}$ be a random variable, and $W\le m$. Moreover, $\E(W)\le m-d$. We have $\E(\log {n \choose m}-\log {n \choose W})=\Omega(d \log n)$.
\end{lemma}

\begin{proof}
  \begin{align}
  \log {n \choose m}-\log {n \choose W}
  &= \log \frac{n!/(m!(n-m)!)}{n!/(W!(n-W)!)} \\
  &= \sum_{i=1}^{m-W}\log \frac{n-W-i+1}{m-i+1} \\
  &\ge (m-W)\cdot \log \frac{n-W}{m} \\
  &\ge (m-W)\cdot \log n^{1/200}
  \end{align}
  
  Taking expectation on both sides, we get $\E(\log {n \choose m}-\log {n \choose W})\ge \frac{d}{200} \log n$. 
\end{proof}

\begin{lemma}\label{lemma:mutual-entropy-vs-fail-prob}
  Let function $f$: $\{0,1\}^n\times \{0,1\}^m\rightarrow \{0,1\}$. Let $X$ be a uniformly random string in $\{0,1\}^n$. If for any $y\in \{0,1\}^m$ we have $\Pr(f(X,y)=1)\le \delta$ where $0<\delta<1$, then for any random variable $Y$ in $\{0,1\}^m$, we have 
  \begin{align}
    \Pr(f(X,Y)=1)\le \frac{I(X;Y)+1}{\log \frac{1}{\delta}},
  \end{align}
  where $I(X;Y)$ is the mutual information (in bits) between $X$ and $Y$.
\end{lemma}

\begin{proof}
  It is equivalent to prove $I(X;Y)\ge \E(f(X,Y))\cdot \log\frac{1}{\delta}-1$. By definition of mutual entropy, $I(X;Y)=H(X)-H(X|Y)$ where $H(X)=n$ and $H(X|Y)\le 1+(1-\E(f(X,Y)))\cdot n+\E(f(X,Y))\cdot (n-\log\frac{1}{\delta})=n+1-\E(f(X,Y))\cdot \log\frac{1}{\delta}$.
  The upper bound for $H(X|Y)$ is obtained by considering the following one-way communication problem: Alice obtains both $X$ and $Y$ while Bob only gets $Y$, what is the (minimum) expected number of bits that Alice sends to Bob so that Bob can recover $X$? 
  Any protocol gives an upper bound for $H(X|Y)$, and we simply take the following protocol: first Alice sends Bob $f(X,Y)$ (taking $1$ bit); and then if $f(X,Y)=0$ Alice sends $X$ directly (taking $n$ bits), otherwise, $f(X,Y)=1$, Alice sends the index of $X$ in $\{x|f(x,Y)=1\}$ (taking $\log (\delta 2^n)=n-\log\frac{1}{\delta}$ bits).  
\end{proof}

\begin{corollary}\label{corollary:sampler-failure}
  Given a sampler with failure probability at most $\delta$ whenever the updates to it are independent from the randomness used by the sampler. Let $X$ denote the random source used by the sampler, and let $D$ denote the random source generating the updates. Then the probability that the sampler fails on updates $D$ is at most $\frac{I(X;D)+1}{\log\frac{1}{\delta}}$.
\end{corollary}

\begin{lemma}\label{lemma:mutual-entropy-bound}
  $I(X;A_r)\le 5K$, for $r=1,\ldots, R$.
\end{lemma}

\begin{proof}
  $I(X;A_r)=H(A_r)-H(A_r|X)$. Since $|A_r|=n_r$ and $A_r\subseteq A$ where $|A|=m$, $H(A_r)\le \log {m \choose n_r}$. Now we want to lower bound $H(A_r|X)$. By definition of conditional entropy, $H(A_r|X)=\sum_x{p_x\cdot H(A_r|X=x)}$. We fix an arbitrary $x$. If we can prove that for any $T\subseteq A$ where $|T|=n_r$, $\Pr(A_r=T|X=x)\le p$, then by definition of entropy we have $H(A_r|X=x)\ge\log\frac{1}{p}$. In fact, for any fixed $T$, we have
  
  \begin{align}
    \Pr(A_r=T|X=x)\le \prod_{i=1}^{r}{\frac{{n_{i-1}-n_r-1 \choose n_{i-1}-n_i-1}}{{n_{i-1}-1 \choose n_{i-1}-n_i-1}}},
  \end{align}
  
  because on round $i$ ($1\le i \le r$), Alice removes $n_{i-1}-n_i$ elements from $A_{i-1}$ to get $A_i$. Conditioned on the event that $A_{i-1}\supseteq T$, the probability that $A_i\supseteq T$ is at most ${{n_{i-1}-n_r-1 \choose n_{i-1}-n_i-1}}/{{n_{i-1}-1 \choose n_{i-1}-n_i-1}}$, where the equation achieves when $s_i\in A_{i-1}\backslash T$, and Alice takes a uniformly random subset of $A_{i-1}\backslash \{s_i\}$ of size $n_{i-1}-n_i-1$, so that the subset does not intersect with $T$.
  
  For notation simplicity, let $n^{\underline{k}}$ denote $n\cdot (n-1)\ldots (n-k+1)$. We have 
  \begin{align}
    \prod_{i=1}^{r}{\frac{{n_{i-1}-n_r-1 \choose n_{i-1}-n_i-1}}{{n_{i-1}-1 \choose n_{i-1}-n_i-1}}}
    =\prod_{i=1}^{r}\frac{(n_{i-1}-n_r-1)!n_i!}{(n_{i-1}-1)!(n_i-n_r)!}
    =\prod_{i=1}^{r}\frac{n_i^{\underline{n_r}}}{(n_{i-1}-1)^{\underline{n_r}}}
    =\prod_{i=1}^{r} \left( \frac{n_i^{\underline{n_r}}}{n_{i-1}^{\underline{n_r}}}\cdot \frac{n_{i-1}}{n_{i-1}-n_r} \right).
  \end{align}
  
  By telescoping,
  \begin{align}
    \prod_{i=1}^{r} \frac{n_i^{\underline{n_r}}}{n_{i-1}^{\underline{n_r}}}
    =\frac{n_r^{\underline{n_r}}}{n_0^{\underline{n_r}}}
    =\frac{n_r!(n_0-n_r)!}{n_0!}=\frac{1}{{n_0 \choose n_r}}
    =\frac{1}{{m \choose n_r}}.
  \end{align}
  
  Moreover, 
  \begin{align}
    \prod_{i=1}^{r} \frac{n_{i-1}}{n_{i-1}-n_r}
    =\prod_{i=1}^{r} \frac{1}{1-2^{(i-1-r)/K}}
    =\prod_{j=1}^{r} \frac{1}{1-2^{-j/K}}
    \le \prod_{j=1}^{\infty} \frac{1}{1-2^{-j/K}}.
  \end{align}
  
  By Lemma~\ref{lemma:Pochhammer}, we have $\prod_{j=1}^{\infty} \frac{1}{1-2^{-j/K}}\le 2^{5K}$. Let $p={2^{5K}}/{{m\choose n_r}}$, we have $\Pr(A_r=T|X=x)\le p$ and thus $H(A_r|X=x)\ge \log\frac{1}{p}=\log{{m\choose n_r}}-5K$. Therefore, $H(A_r|X)\ge \log{{m\choose n_r}}-5K$ and so $I(X;A_r)=H(A_r)-H(A_r|X)\le 5K$.  
\end{proof}

%By \url{http://mathworld.wolfram.com/q-PochhammerSymbol.html} We have $\prod_{j=1}^{\infty} \frac{1}{1-2^{-j/K}}\le 2^{5K}$. I think we can improve the constant 5 to 4 if we bound it more carefully. 
\begin{lemma}\label{lemma:Pochhammer}
  Let $K\in \mathbb{N}$ and $K\ge 1$. We have $\prod_{j=1}^{\infty} \frac{1}{1-2^{-j/K}}\le 2^{5K}$.
\end{lemma}

\begin{proof}
  First, we bound the product of first $2K$ terms. Note that $\frac{1}{1-2^{-x}}\le \frac{8}{3x}$ for $0<x\le 2$. Therefore, 
  \begin{align}
    \prod_{j=1}^{2K}\frac{1}{1-2^{-j/K}}
    \le (8/3)^{2K}\cdot \frac{K^{2K}}{(2K)!}
    \le (8/3)^{2K}\cdot \frac{K^{2K}}{(2K/e)^{2K}}
    = (4e/3)^{2K}
    < 2^{4K}. 
  \end{align}
  
  Then, we bound the product of the rest terms
  \begin{align}
    \prod_{j=2K+1}^{\infty}\frac{1}{1-2^{-j/K}} 
    \le \prod_{j=2K+1}^{\infty}\frac{1}{1-2^{-\lfloor j/K \rfloor}} 
    \le \prod_{i=2}^{\infty}\left( \frac{1}{1-2^{-i}}\right)^K 
    \le \left( \frac{1}{1-\sum_{i=2}^{\infty}2^{-i}}\right)^K
    = 2^K.
  \end{align}
  
  Multiplying two parts proves the lemma.
\end{proof}

\begin{theorem}
  $\s = \Omega(\log^2 n\log{\frac{1}{\delta}})$.
\end{theorem}

\begin{proof}
  By Lemma~\ref{lemma:zero-fail-prob}, the success probability of protocol $(\enc,\dec)$ is $1$. 
  By Lemma~\ref{lemma:lb-meta}, we have $\s\ge \log (^n_m) - \s' -2$, where $\s'=\log n + R+ \E(\log (^n_{|B|}))$. 
  The size of $B$ is $|B|=|A|-\sum_{r=1}^{R}{b_r}$.
  By Corollary~\ref{corollary:sampler-failure}, and because conditioned on $A$, the randomness of updates to $\samp_r$ comes from $A_{r-1}$ (for $r=1, \ldots, R$), we get $\Pr(b_r=0)\le \frac{I(X;A_{r-1})+1}{\log\frac{1}{\delta}}$ and thus $\E(b_r)\ge 1-\frac{I(X;A_{r-1})+1}{\log\frac{1}{\delta}}$. 
  By Lemma~\ref{lemma:mutual-entropy-bound}, $I(X;A_{r-1})\le 5K$ (Note that when $r=1$, $I(X;A_0)=0\le 5K$). Therefore, $\E(b_r)\ge 1-\frac{5K+1}{\log\frac{1}{\delta}}$.
  By the setting of parameters (see Algorithm~\ref{algo:para}) we have $\E(b_r)\ge \frac{2}{5}$. Therefore, $\E(|B|)\le |A|-\frac{2}{5}R$. 
  By Lemma~\ref{lemma:bits-saving}, $\E(\log (^n_{|B|}))=\log (^n_m)-\Omega(R\log n)$. 
  Combining together we get $\s=\Omega(R\log n)=\Omega(\log^2 n \log\frac{1}{\delta})$.
\end{proof}